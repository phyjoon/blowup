\documentclass[letterpaper, 11pt]{article}

\usepackage{jheppub}
\usepackage{bm}
\usepackage{graphicx}
\usepackage{epstopdf}
\usepackage{amsmath, amssymb}

\usepackage{amsfonts,amssymb,epsfig,amsmath,mathtools,tabu}
\usepackage{verbatim}

%\usepackage[utf8]{inputenc}
%\usepackage{textcomp,setspace}
\usepackage[inline]{enumitem}

\usepackage{hyperref,caption,subcaption}
\usepackage{xcolor,tikz,graphicx,afterpage}
\usetikzlibrary{shapes.geometric,positioning}
%\usetikzlibrary{cd}


\newcommand{\be}{\begin{eqnarray}}
\newcommand{\ee}{\end{eqnarray}}
\newcommand{\nn}{\nonumber}
\newcommand{\bn}{\begin{enumerate}}
\newcommand{\en}{\end{enumerate}}

%%%%%%%%%%%%%%%%%%% Figures %%%%%%%%%%%%%%%%%%%%%%%%%%%

\newcommand{\fig}[3]{
\begin{figure}
\centerline{\epsfxsize=#1\epsfbox{#2.eps}}
\newcaption{#3. \label{#2}}
\end{figure}
}

%%%%%%%%%%%%% Double line letters using amssymb %%%%%%%%%%%%%%%%

\def\identity{{\rlap{1} \hskip 1.6pt \hbox{1}}}
\def\iden{\identity}

\def\IB{\mathbb{B}}
\def\IC{\mathbb{C}}
\def\ID{\mathbb{D}}
\def\IH{\mathbb{H}}
\def\IM{\mathbb{M}}
\def\IN{\mathbb{N}}
\def\IP{\mathbb{P}}
\def\IR{\mathbb{R}}
\def\IZ{\mathbb{Z}}

%%%%%%%%%%%%%%%% Caligraphic letters %%%%%%%%%%%%%%%%%%

\def\CA{{\cal A}}
\def\CB{{\cal B}}
\def\CC{{\cal C}}
\def\CD{{\cal D}}
\def\CE{{\cal E}}
\def\CF{{\cal F}}
\def\CG{{\cal G}}
\def\CH{{\cal H}}
\def\CI{{\cal I}}
\def\CJ{{\cal J}}
\def\CK{{\cal K}}
\def\CL{{\cal L}}
\def\CM{{\cal M}}
\def\CN{{\cal N}}
\def\CO{{\cal O}}
\def\CP{{\cal P}}
\def\CQ{{\cal Q}}
\def\CR{{\cal R}}
\def\CS{{\cal S}}
\def\CT{{\cal T}}
\def\CU{{\cal U}}
\def\CV{{\cal V}}
\def\CW{{\cal W}}
\def\CX{{\cal X}}
\def\CY{{\cal Y}}
\def\CZ{{\cal Z}}

%%%%%%%%%%%%%%%%%% Greek letters %%%%%%%%%%%%%%%%%%%%%%%%%%%%

\def\a{\alpha}
\def\b{\beta}
\def\g{\gamma}
\def\d{\delta}
\def\e{\epsilon}
\def\ve{\varepsilon}
\def\z{\zeta}
% eta
\def\th{\theta}
\def\vth{\vartheta}
\def\i{\iota}
\def\k{\kappa}
\def\l{\lambda}
\def\m{\mu}
\def\n{\nu}
% xi
% o
% pi
\def\vp{\varpi}
\def\r{\rho}
\def\vr{\varrho}
\def\s{\sigma}
\def\vs{\varsigma}
\def\t{\tau}
\def\u{\upsilon}
% phi
\def\vph{\varphi}
% chi
\def\ch{\chi}
% psi
\def\w{\omega}
%
\def\G{\Gamma}
\def\D{\Delta}
\def\Th{\Theta}
\def\L{\Lambda}
% Xi
% Pi
\def\S{\Sigma}
\def\Y{\Upsilon}
% Phi
% Psi
\def\O{\Omega}


%%%%%%%%%%%%%%%%% Mathematical Symbols %%%%%%%%%%%%%%%%%%%%%%%%%%%%

\def\half{\frac{1}{2}}
\def\thalf{{\textstyle \frac{1}{2}}}
\def\imp{\Longrightarrow}
\def\goto{\rightarrow}
\def\para{\parallel}
\def\vev#1{\langle #1 \rangle}
\def\del{\nabla}
\def\grad{\nabla}
\def\curl{\nabla\times}
\def\div{\nabla\cdot}
\def\p{\partial}
\newcommand{\bra}[1]{\langle{#1}|}
\newcommand{\ket}[1]{|{#1}\rangle}
\def\fslash{\displaystyle{\not}}

%%%%%%%%%%%%%%%%%%%% Normal font in math %%%%%%%%%%%%%%%%%%%%%%%%%%

\def\Tr{{\rm Tr}}
\def\tr{{\rm tr}}
\def\det{{\rm det}}



%%%%%%%%%%%%%%%%%%%%%%%%%%%%%%%%%%%%%%%%%%%%%%%%%%%%%
\title{Exceptional Instantons from Blow-up}

\author[a]{Joonho Kim,}
\author[b]{Sung-Soo Kim,}
\author[c]{Ki-Hong Lee,}
\author[a]{Kimyeong Lee,}
\author[a]{and Jaewon Song}
\affiliation[a]{School of Physics, Korea Institute for Advanced Study, Seoul 02455, Korea}
\affiliation[b]{School of Physics, University of Electronic Science and Technology of China,\\ No.4, Section 2, North Jianshe Road, Chengdu, Sichuan 610054, China}
\affiliation[c]{Department of Physics and Astronomy \& Center for Theoretical Physics\\ Seoul National University, Seoul 08826, Korea}
\emailAdd{joonhokim@kias.re.kr}
\emailAdd{sungsoo.kim@uestc.edu.cn}
\emailAdd{khlee11812@gmail.com}
\emailAdd{klee@kias.re.kr}
\emailAdd{jsong@kias.re.kr}

\abstract{
The Nekrasov partition function for 4d $\CN=2$ or 5d $\CN=1$ gauge theory on the blow up of a point $\hat{\IC}^2$ can be written in terms of the partition function on the flat space $\IC^2$. At the same time, the partition function on the blow up is identical to the partition function on a flat space for sufficiently large class of examples. 
This relation enables us to compute the instanton partition functions for 4d $\CN=2$ and 5d $\CN=1$ gauge theories for arbitrary gauge theory with large class of matter representations. 
Especially, we obtain the partition function for the exceptional gauge groups $EFG$ with fundamental hypermultiplets/spinors and more. We also compute the case with $SU(6)$ with rank-3 antisymmetric tensor and compare with the topological vertex computation using the recently found 5-brane web configuration. 
}


\preprint{KIAS-P19???, SNUTP19-???}

%%%%%%%%%%%%%%%%%%%%%%%%%%%%%%%%%%%%%%%%%%%%%%%%%%%
\begin{document}
\maketitle

\section{Introduction} \label{sec:intro}

%%%%%%%%%%%%%%%%%%%%%%%%%%%%%%%%%%%%%%%%%%%%%%%%%%%%%%%%%%%%%%%%

\section{Instanton Counting from Blow-up} \label{sec:blowup}

The essential idea of using the blow-up for the instanton counting is that the gauge theory partition function for a 4d $\CN=2$ theory or 5d $\CN=1$ on a blow up of a point $\hat{\IC}^2$ (or $S^1 \times \hat{\IC}^2$) can be written in two different ways. This allows us to write a recursion relation for the instanton partiton function that can be solved easily \cite{Nakajima:2003pg, Nakajima:2003uh,Nakajima:2005fg, Keller:2012da}. 

One of the expression is obtained via localization on the Coulomb branch obtained by patching together flat-space partition functions. The blow up of a point of $\IC^2$ can be described as a subspace of $\IC^2 \times \IP^1$ defined as
\begin{align}
 \{ (x, y), [z: w] \in \IC^2 \times \IP^1 | xw = yz \} \ , 
\end{align}
where $[z:w]$ represents the homogeneous coordinates on $\IP^1$. Notice that this space is identical to $\IC^2$ when $(x, y) \neq (0, 0)$, and the origin is replaced by complex projective plane. We will be considering $U(1)^2$-equivariant partition function, with each $U(1)$'s acting as the rotation of two complex planes. This $U(1)^2$ action acts on $\hat{\IC}^2$ as
\begin{align}
 ((x, y), [z: w]) \mapsto (e^{i \e_1} x, e^{i \e_2} y), [e^{i \e_1} z, e^{i \e_2} w]) \ . 
\end{align}
The zero sized instantons will be located at the two fixed points of the $U(1)^2$ action, namely the north pole and the south pole of the sphere. At these points, the weights of the $U(1)^2$ action become
\begin{align}
 ((0, 0), [1, 0]) : (\e_1, \e_2 - \e_1) , \qquad ((0, 0), [0, 1]): (\e_1 - \e_2, \e_2) \ . 
\end{align}

Now, the (full) partition function on a blow up (includes both the perturbative and the instanton parts) $\hat{Z}$ can be written as a sum over a product of the partition functions at two fixed points as (here we turn off any external flux that can be supported on the blow up) [References]
\begin{align} \label{eq:blowup}
 \hat{Z}(\vec{a}, \e_1, \e_2) = \sum_{\vec{k} \in \Lambda} Z(\vec{a}+ \vec{k} \e_1, \e_1, \e_2 - \e_1) Z(\vec{a}+\vec{k} \e_2, \e_1 - \e_2, \e_2) \ , 
\end{align}
where $Z(\vec{a}, \e_1, \e_2)$ is the (full) partition function on $\IC^2$ where $\vec{a}$ is the Coulomb parameter. 
Here $\Lambda$ is the weight lattice of the gauge group and the vector $\vec{k}$ labels different flux configurations on the divisor of the blow-up classified by the first Chern numbers. 
We obtain this expression by performing the localization on the Coulomb branch. On the Coulomb branch, the gauge group is broken to $U(1)^r$ where $r$ is the rank of the gauge group. We need to patch together all possible field configurations with zero sized instantons located at north and south poles, and all the inequivalent configurations are labeled by the first Chern numbers which we have to sum over. The shift of the Coulomb parameter comes from the gauge transformation along the equator that connects the north and the south pole so that $\vec{a}_N - \vec{a}_S = \vec{k} (\e_1 - \e_2)$. 


The second expression for the partition function on the blow-up is that $\hat{Z}$ is actually identical to the partition function $Z$ on the flat space $\IC^2$ (or $S^1 \times \IC^2$) \cite{Nakajima:2003pg, Nakajima:2003uh,Nakajima:2005fg}. This can be argued as follows: The blow up $\hat{\IC}^2$ is identical to $\IC^2$ except for the origin, which is replaced by $\IP^1$, called as the exceptional divisor. The blow-up can be also identified as the total space of the bundle $\CO(-1) \to \IP^1$. 
The Nekrasov partition function gets contributions only from the zero-size instantons and the finite size of the divisor should not affect the partition function as we smoothly shrink the size. Therefore, we expect that the partition function on the blow-up to be identical to that of the flat $\IC^2$. (In some sense, the partition function is a birational invariant. But there can be wall-crossing.)

This is a special feature of the blow-up of a point. In principle, there can be extra states coming from the $\IP^1$. As we shrink the size of the $\IP^1$, they may become massless which might in principle contribute to the partition function when there is a singularity as we blow-down. 
For example, if we consider the total space of $\CO(-2) \to \IP^1$ and shrink the base, we land on the orbifold $\IC^2/\IZ_2$ which is singular at the origin. In our case, we do not obtain any singularity as we blow down the sphere.  As we will discuss later, this simple picture does not necessarily hold when there are too many hypermultiplets. But this is actually shown to be the case for the pure YM theory, and we will also demonstrate that it is true for many interesting cases.\footnote{The orbifold partition function can be also obtained in two different ways modulo some subtle scheme dependence related to wall-crossing.}

Therefore, we have a powerful identity:
\begin{align}
 Z(\vec{a}, \e_1, \e_2) = \sum_{\vec{k} \in \Lambda} Z(\vec{a}+ \vec{k} \e_1, \e_1, \e_2 - \e_1) Z(\vec{a}+\vec{k} \e_2, \e_1 - \e_2, \e_2) \ . 
\end{align}
But this is not enough to determine the partition function since there are three unknown functions and one relation. To achieve this we need to consider a `correlation function' $\langle \mu(C) \mu(C) \ldots \rangle$ of operators $\mu$ associated to the two-cycle in the blow-up so that we get 
\begin{align}
 \hat{Z}^{(d)}(\vec{a}, m, \e_1, \e_2) = \sum_{\vec{k} \in \Lambda} \mu(\vec{k})^d Z(\vec{a}+ \vec{k} \e_1, \e_1, \e_2 - \e_1) Z(\vec{a}+\vec{k} \e_2, \e_1 - \e_2, \e_2) \ . 
\end{align} 
We can insert arbitrary number of $\mu$'s and they in turn can be written in a very simple form. Now, as we shrink the two cycles to recover the flat $\IC^2$, the effect of the insertion of $\mu$ turns out to give either a trivial or vanishing contribution due to the conservation laws. This allows us to write 3 relations among for the 3 unknowns to solve for the instanton partition function. 

In the remainder of this section we will give detailed formulae for 4d and 5d gauge theories. 

\subsection{4d gauge theory}
Building blocks for the 4d
\begin{align}
 Z_{\textrm{vec}}^{\textrm{pert}}(\vec{a}, q) &= \exp \left( - \sum_{\vec \a \in \Delta} \gamma_{\e_1, \e_2}  (\vec{a} \cdot {\vec{\a}}; q )\right) \\
 Z_{\textrm{hyp}}^{\textrm{pert}} (\vec{a}, m, q)&= \exp \left( \sum_{\vec{w} \in R} \gamma_{\e_1, \e_2} ( \vec{a} \cdot {\vec{w}} - m; q)\right)
\end{align}
Here the gamma function is defined as
\begin{align}
 \gamma_{\e_1, \e_2} (x; \Lambda) = \left. \frac{d}{ds} \right|_{s=0} \frac{\Lambda^s}{\Gamma(s)} \int_0^{\infty} \frac{dt}{t} t^s \frac{e^{-ts}}{(e^{\e_1 t} - 1)(e^{\e_2 t} - 1)} \ , 
\end{align}
which is formally equivalent to 
\begin{align}
 \textrm{log} \left[\prod_{n, m\ge 0} \left( \frac{x - m\e_1 - n \e_2}{\Lambda} \right) \right] \ . 
\end{align}

If we write the equation \eqref{eq:blowup} in terms of $Z = Z^{\textrm{pert}} Z^{\textrm{inst}}$, we get
\begin{align}
 Z^{\textrm{inst}}(\vec{a}, \e_1, \e_2) = \sum_{\vec{k} \in \Lambda} \frac{Z^{(N), \textrm{pert}}(\vec{k}) Z^{(S), \textrm{pert}}(\vec{k})}{Z^{\textrm{pert}}(\vec{a}, \e_1, \e_2)} Z^{(N), \textrm{inst}}(\vec{k} ) Z^{(S), \textrm{inst}}(\vec{k})
\end{align}
where
\begin{align}
 Z^{(N)} (\vec{k}) &= Z(\vec{a}+ \e_1 \vec{k}, \e_1, \e_2 - \e_1) \\
 Z^{(S)} (\vec{k}) &= Z(\vec{a} + \e_2 \vec{k}, \e_1 - \e_2, \e_2)
\end{align}
and here we omit the dependence on the Coulomb vev and the Omega deformation parameters. 
Now the factor in the middle can be explicitly worked out. Let us denote the ratio of the perturbative factor as $f(\vec{k}) \equiv Z^{(N), \textrm{pert}} Z^{(S), \textrm{pert}}/Z^{\textrm{pert}}$. 
%The tree level part can be worked out to get
%\begin{align}
% f(\vec{k})^{\textrm{tree}} = q^{ - \frac{(a+k\e_1)^2}{\e_1 (\e_2 - \e_1)} - \frac{(a+k\e_2)^2}{\e_1 (\e_2 - \e_1)}+ \frac{a^2}{\e_1 \e_2} } = q^{k^2}
%\end{align}
Then the ratio of the perturbative factor for the vector multiplet is given as
\begin{align} \label{eq:1loopvec}
f(\vec{k})_{\textrm{vec}} &= \prod_{\a \in \Delta} \exp \left( \g_{\e_1, \e_2} (\vec{a}\cdot \vec{\a}) - \g_{\e_1, \e_2 - \e_1}(\vec{a}\cdot \vec{\a} + \vec{k}\cdot\vec{\a} \e_1) -  \g_{\e_1 - \e_2, \e_2 }(\vec{a}\cdot \vec{\a} +  \vec{k}\cdot\vec{\a} \e_2)   \right) \\
 &= \prod_{\vec{\a} \in \Delta}  \frac{\Lambda^{(\vec{k} \cdot \vec{\a})^2 /2} }{s(-\vec{k}\cdot \vec{\a}, \vec{\a}\cdot \vec{a}, \e_1, \e_2) }
 = \frac{(\Lambda^{2 h^\vee})^{\vec{k} \cdot \vec{k}/2} }{\prod_{\vec{\a} \in \Delta} \ell^{\vec{k}}_{\vec{\a}} (\vec{a}, \e_1, \e_2) }
\end{align}
where $h^\vee)$ refers to the dual Coxeter number of the gauge group. Notice that ithe beta function coefficient for the pure YM theory is given by $b_0 = 2h^\vee$ and the instanton number is given as $q \equiv \Lambda^{2 h^\vee}$. The other symbols are given as 
\begin{align}
\ell^{\vec{k}}_{\vec{\alpha}} (\vec{a}, \e_1, \e_2) &= s(-\vec{k}\cdot\vec{\a}, \vec{a}\cdot\vec{\a}, \e_1, \e_2) \\
s(k, x, \e_1, \e_2) &= 
\begin{cases}
 {\displaystyle \prod_{i, j \ge 0, i+j \le k-1} (x - i \e_1 - j e_2) } & (k > 0) \\
 {\displaystyle \prod_{i, j \ge 0, i+j \le -k-2} (x + (i+1)\e_1 + (j+1)\e_2)} & (k < -1)  \\
 1 & (k=0, -1)
\end{cases}
\end{align}
The final identity of \eqref{eq:1loopvec} involves a bit of work. This follows from the identity (\cite{Nakajima:2003uh}, App. E.)
\begin{align}
 \g_{\e_1, \e_2-\e_1}(x+\e_1 k; \Lambda) + \g_{\e_1 - \e_2, \e_2}(x+\e_2k; \Lambda)
  = \g_{\e_1, \e_2}(x; \Lambda) + \log s(-k, x, \e_1, \e_2) - \frac{k(k-1)}{2} \log \Lambda \ . 
\end{align}
For the hypermultiplets we get, 
\begin{align}
f(\vec{k})_{\textrm{hyp}} &=  \prod_{\vec{w} \in R} \exp \Big( -\g_{\e_1, \e_2} (a_w - m) + \g_{\e_1, \e_2 - \e_1}(a_w + k_w \e_1 - m) +  \g_{\e_1 - \e_2, \e_2 }(a_w + k_w \e_2 - m)   \Big) \nn \\
&= \prod_{\vec{w} \in R} \Lambda^{-\half k_w^2} s(-{k}_w, {a}_w - m, \e_1, \e_2) 
= (\Lambda^{- 2 C_2(R)})^{\vec{k} \cdot \vec{k}/2 }\prod_{\vec{w} \in R}  s(-{k}_w, {a}_w - m, \e_1, \e_2) 
\ , 
\end{align}
where we introduced the short-hand notation $k_w = \vec{k}\cdot\vec{w}$, $a_w = \vec{a} \cdot \vec{w}$ and $C_2(R)$ corresponds to the second Casimir invariant for the representation $R$. This invariant appears in the beta function coefficients as $b_0 = 2h^\vee - 2 \sum_R C_2(R)$ for the hypermultiplets in irrep $R$.  This gives the instanton parameter to be $q\equiv \Lambda^{b_0} = \Lambda^{2h^\vee - 2 \sum_R C_2(R)}$. 

Now, for the SQCD, we obtain the following equation:
\begin{align}
  Z^{\textrm{inst}}(\vec{a}, m, \e_1, \e_2) = \sum_{\vec{k} \in \Lambda} f(\vec{k}) Z^{(N), \textrm{inst}}(\vec{k} ) Z^{(S), \textrm{inst}}(\vec{k}) \ , 
\end{align}
with
\begin{align}
 f(\vec{k}) =  \frac{\displaystyle q^{\half \vec{k} \cdot \vec{k}} \prod_{i} \prod_{\vec{w} \in R_i}  s(-k_w, a_w - m_i, \e_1, \e_2)}{\displaystyle \prod_{\vec{\a} \in \Lambda} s(-k_\a, a_\a, \e_1, \e_2)} \ , 
\end{align} 
where $i$ runs over the charged hypermultiplets. Here $q = \Lambda^{2N_c - N_f}$ for the $SU(N_c)$ SQCD with $N_f$ fundamental hypermultiplets. 
We have checked this expression explicitly for the $SU(2)$ gauge theory with $N_f=0, 1$ hypermultiplets up to the first few order in instanton numbers. 

Notice that in the Gottsche-Nakajima-Yoshioka \cite{Nakajima:2009qjc, Gottsche:2010ig}, the mass parameters and the instanton parameters (for the 5d) are also shifted when the contribution from North and South poles are computed. This is simply a reflection of the fact that they twist the instanton bundles by the half-Canonical bundle of the $\IC^2$, which shift the mass parameters by $m \to m - \frac{\e_1 + \e_2}{2}$. If we do not twist by this amount, we get a cleaner expression as above. 

\paragraph{Correlation functions}
On the blow-up, we have a non-trivial 2-cycle. This allows us to insert the equivariant version of the Donaldson operator, which can be written in terms of the twisted fields as
\begin{align} \label{eq:muC}
 \left\langle \exp \left[ t \int d^4 x \left( \omega \wedge \phi F + \half \psi \wedge \psi + H F \wedge F \right) \right] \right\rangle \ . 
\end{align}
Here $H$ is the moment map for the $U(1)^2_{\e_1, \e_2}$ action so that $dH = \i_{V} \omega$. Upon localization, it is easy to see that it can be written as
\begin{align}
 \hat{Z}^{\textrm{inst}}(\vec{a}, m, \e_1, \e_2, t) = \sum_{\vec{k} \in \Lambda} e^{ t \left( \vec{k} \cdot \vec{a} + \half \vec{k}\cdot \vec{k} (\e_1 + \e_2) \right) } f(\vec{k}) Z^{\textrm{inst}}(\vec{k}; q e^{t \e_1} ) Z^{(S), \textrm{inst}}(\vec{k} ; q e^{t \e_2}) \ . 
\end{align} 
Notice that we shift the instanton parameter by equivariant parameters. This is due to the term $H F\wedge F$ in \eqref{eq:muC} which shifts the instanton parameter by $e^{t(\e_1+\e_2)}$. We need to compensate this part by shifting $q$. 
The insertion \eqref{eq:muC} inside the exponent has the conserved charge $U=+2$, and the instanton generates $4h^\vee$ which means it counts the number of fermion zero modes. Therefore, when expanding in powers of $t$, 
\begin{align}
\hat{Z} = Z + \CO(t^{2h^\vee - 2\sum_R C_2(R)})  \ . 
\end{align}
Therefore, when $N_f < 2N_c - 2$ for the $SU(N_c)$ SQCD, we have enough number of relations to fix the instanton partition function. 


\subsection{Blowup Equation for 5d gauge theory} 

We consider a 5d $\mathcal{N}=1$ gauge theory with a gauge group $G$. It has the Coulomb branch moduli space, parametrized by the vacuum expectation value $\alpha_i\equiv \langle\Phi_{ii}\rangle $ of the vector multiplet scalar $\Phi$. The gauge symmetry $G$ is spontaeously broken to its Abelian subgroup $U(1)^{|G|}$ on the Coulomb branch. The low energy Abelian theory is described by the effective prepotential, which can be written as \cite{Intriligator:1997pq}
\begin{align}
	\mathcal{F}_\text{classical} = \frac{1}{2g^2}h_{ij}\alpha_i \alpha_j + \frac{\kappa}{6}d_{ijk}\alpha_i\alpha_j\alpha_k 
\end{align}
at the classical level. It was found in \cite{Nekrasov:2002qd,Nekrasov:2003rj} that the fully quantum corrected prepotential can be obtained from the BPS instanton partition function $\mathcal{Z}$ on $\Omega$-deformed $\mathbf{R}^4 \times S^1$, i.e.,
\begin{align}
	\mathcal{F} = \lim_{\epsilon_{1,2}\rightarrow 0} \epsilon_1 \epsilon_2 \log{\mathcal{Z}} = \mathcal{F}_\text{classical} + \mathcal{F}_\text{quantum}.
\end{align}

The BPS partition function $\mathcal{Z}$ is defined with equivariant parameters $\epsilon_{1}, \epsilon_2$, $\alpha_1, \cdots, \alpha_{|G|}$ associated to the $U(1)^2 \times U(1)^{|G|}$ action on the $k$-instanton moduli space $\mathcal{M}_{k,G}$  \cite{Nekrasov:2002qd,Nekrasov:2003rj}. Additional equivariant parameters $m_1, \cdots, m_{|F|}$ can be introduced if a given theory has the flavor symmetry $F$. It takes the form of 
\begin{align}
    \mathcal{Z} = \exp{(F_0)} \cdot \mathcal{I}
\end{align}
where $\mathcal{I}$ is the Witten index counting the BPS bound states of fundamental particles and/or non-perturbative instanton solitons. More precisely, 
\begin{align}
	\mathcal{I} \equiv \text{Tr}_{\mathcal{H}} \bigg[(-1)^F e^{-\frac{8\pi^2}{g^2} k} e^{-\epsilon_1 (J_1+\frac{R}{2})} e^{-\epsilon_2 (J_2+\frac{R}{2})} \prod_{i=1}^{|G|} e^{-\alpha_i Q_i} \prod_{l=1}^{|F|} e^{-m_l F_l}\bigg]
\end{align}
where $(J_1, J_2)$ are the angular momenta associated to the two $\mathbf{R}^2$ planes, $R$ is the Cartan generator of $SU(2)_R$ symmetry, $(Q_1, \cdots, Q_{|G|})$ are the electric charges of $U(1)^{|G|} \subset G$, and $(F_1, \cdots, F_{|F|})$ are the Cartan generators of the flavor symmetry group $F$. 
We also frequently use the notation $\epsilon_+ \equiv \frac{\epsilon_1 + \epsilon_2}{2}$, $\epsilon_- \equiv \frac{\epsilon_1 - \epsilon_2}{2}$ and $J_l = \frac{J_1 - J_2}{2}$, $J_r = \frac{J_1 + J_2}{2}$, generating self-dual and anti-self-dual rotation inside the $\mathbf{R}^4$.
The fugacity variables used throughout this paper are 
\begin{align}
    p_1 = e^{-\epsilon_1},\ p_2 = e^{-\epsilon_2},\ \omega_i = e^{-\alpha_i},\ y_l = e^{-m_l},\ Q = e^{-8\pi^2  /g^2},\ t = \sqrt{p_1p_2},\  u = \sqrt{p_1/p_2}.
\end{align}

Each multiplet




%%%%%%%%%%%%%%%%%%%%%%%%%%%%%%%%%%%%%%%%%%%%%%%%%%%%%%%%%%%%%%%%

\section{Examples} \label{sec:example}

\subsection{Ki-Hong's note}
\paragraph{Unity Blowup equations}
The partition functions of generic 5d $\mathcal{N}=1$ gauge theories with hypermultiplets in $R$-representation in the Coulomb branch consist of classical action term, 1-loop term, and instanton partition functions.
\begin{align}
Z(\epsilon_1,\epsilon_2,\vec{a},m_i,m_0)=Z_{\textrm{class}}(\epsilon_1,\epsilon_2,\vec{a},m_0)\,Z_{\textrm{1-loop}}(\epsilon_1,\epsilon_2,\vec{a},m_i)\,Z_{\textrm{inst}}(\epsilon_1,\epsilon_2,\vec{a},m_i,m_0)
\label{eq:bueq_gen}
\end{align}
where
\begin{align}
Z_{\textrm{class}}=\textrm{exp}\Bigg[&\,-\frac{1}{\epsilon_1\epsilon_2}\left(\frac{1}{2}h_{ij}\phi^i\phi^j+\frac{1}{6}d_{ijk}\phi^{i}\phi^j\phi^k\right)\Bigg]\nonumber\\
Z_{\textrm{1-loop}}=\textrm{exp}\Bigg[&\,-\frac{1}{2\epsilon_1\epsilon_2}\Bigg(\sum_{\alpha\in\textrm{roots}}\Bigg(\frac{1}{6}(\vec{a}\cdot\vec{\alpha})^3-\frac{1}{4}(\epsilon_1+\epsilon_2)(\vec{a}\cdot\vec{\alpha})^2+\frac{1}{12}((\epsilon_1+\epsilon_2)^2+\epsilon_1\epsilon_2)(\vec{a}\cdot\vec{\alpha})\Bigg)\nonumber\\
&\,+\sum_{\omega\in\rho(R)}\Bigg(\frac{1}{6}\Big(\vec{a}\cdot\vec{\omega}+m_i+\frac{\epsilon_1+\epsilon_2}{2}\Big)^3-\frac{\epsilon_1+\epsilon_2}{4}\Big(\vec{a}\cdot\vec{\omega}+m_i+\frac{\epsilon_1+\epsilon_2}{2}\Big)^2\nonumber\\
&\,\qquad\qquad-\frac{(\epsilon_1+\epsilon_2)^2+\epsilon_1\epsilon_2}{24}\Big(\vec{a}\cdot\vec{\omega}+m_i+\frac{\epsilon_1+\epsilon_2}{2}\Big)\Bigg)\Bigg]\nonumber\\
\times\,\textrm{PE}\Bigg[&\,\frac{1}{(1-p_1)(1-p_2)}\Bigg(-\sum_{\alpha\in\textrm{roots}}e^{\vec{a}\cdot\vec{\alpha}}+p_1^{1/2}p_2^{1/2}y_i\sum_{\omega\in \rho(R)}e^{\vec{a}\cdot\vec{\omega}}\Bigg)\Bigg].
\end{align}
Here $\vec{a}$ are Coulomb VEVs and $p_{1,2}=e^{\epsilon_{1,2}}$, $y_i=e^{m_i}$. Note that the normal exponential term saturates the zero-point energy of pletheystic exponential terms.\footnote{Technically, instead of considering this 1-loop prepotential terms, I inserted overall factors to the $l_{\vec{k}}=Z_{\textrm{1-loop}}^{(1)}Z_{\textrm{1-loop}}^{(1)}/Z_{\textrm{1-loop}}$ so that it is written by Sinh terms.} 

The partition function satisfies so-called ``Unity blowup equation" 
\begin{align}
Z(\epsilon_1,\epsilon_2,\vec{a},m_i, m_0)=\sum_{\vec{k}\in\vec{\alpha}^{\lor}}&\,Z(\epsilon_1,\epsilon_2-\epsilon_1,\vec{a}+(\vec{k}+\vec{r}_a)\,\epsilon_1,m_i+r_i\,\epsilon_1,m_0+r_0\,\epsilon_1)\nonumber\\
&\times\, Z(\epsilon_1-\epsilon_2,\epsilon_2,\vec{a}+(\vec{k}+\vec{r}_a)\,\epsilon_2,m_i+r_i\,\epsilon_2,m_0+r_0\,\epsilon_2)
\end{align}
for certain $\vec{r}_a,\, r_i,\,r_0$'s.
Here $\vec{\alpha}^{\lor}$ is the coroot lattice where the long root is normalized to have norm 2. The $r_i$'s and $r_0$ are some numbers specifying the blowup equations.

Technically $r_i$'s are constrained to be half integers since, for each single letter 1-loop partition functions
\begin{align}
Z_{i,\vec{\omega}}=\textrm{PE}\Bigg[\frac{p_1^{1/2}p_2^{1/2}}{(1-p_1)(1-p_2)}y_i\,e^{\vec{a}\cdot\vec{\omega}}\Bigg],
\end{align}
the ratio between shifted ones and unshifted one is
\begin{align}
l^{\vec{k}}_{i,\vec{\omega}}=&\,Z^{(1)}_{i,\vec{\omega}}Z^{(2)}_{i,\vec{\omega}}/Z_{i,\vec{\omega}}\nonumber\\
=&\,\textrm{PE}\Bigg[\frac{p_1^{r_i}p_2^{1/2}y_i}{(1-p_1)(1-p_2/p_1)}p_1^{\vec{k}\cdot\vec{\omega}}e^{\vec{a}\cdot\vec{\omega}}+\frac{p_1^{1/2}p_2^{r_i}y_i}{(1-p_1/p_2)(1-p_2)}p_2^{\vec{k}\cdot\vec{\omega}}e^{\vec{a}\cdot\vec{\omega}}-\frac{p_1^{1/2}p_2^{1/2}y_i}{(1-p_1)(1-p_2)}e^{\vec{a}\cdot\vec{\omega}}\Bigg]\nonumber\\
=&\,\textrm{PE}\Bigg[\frac{p_1^{1/2}p_2^{1/2}y_i}{(1-p_1)(1-p_2)(p_1-p_2)}e^{\vec{a}\cdot\vec{\omega}}\Big((1-p_2)p_1^{\vec{k}\cdot\vec{\omega}+r_i+1/2}-(1-p_1)p_2^{\vec{k}\cdot\vec{\omega}+r_i+1/2}\Big)-(p_1-p_2)\Bigg]
\end{align}
For the $l^{\vec{k}}_{i,\vec{\omega}}$ to be finite rational function, the plethystic exponent must be finite series. It can be satisfied only when $r_i$ is a half integer.

\paragraph{Instanton partition functions from blowup equations}
From blowup equations one can compute the partition functions as follows. Rewriting the blowup equation as
\begin{align}
1=\sum_{\vec{k}\in\vec{\alpha}^{\lor}}f_{\vec{k}}\,l_{\vec{k}}\frac{Z^{(1)}_{\textrm{inst}}Z^{(2)}_{\textrm{inst}}}{Z_{\textrm{inst}}}
\end{align}
where $f_{\vec{k}}=Z^{(1)}_{\textrm{class}}Z^{(2)}_{\textrm{class}}/Z_{\textrm{class}}$ and $l_{\vec{k}}=Z^{(1)}_{\textrm{1-loop}}Z^{(2)}_{\textrm{1-loop}}/Z_{\textrm{1-loop}}$ with abbreviated notation 
\begin{align}
Z^{(1)}=Z(\epsilon_1,\epsilon_2-\epsilon_1,\vec{a}+\vec{k}\,\epsilon_1,m_i+r_i\,\epsilon_1,m_0+r_0\,\epsilon_1)\nonumber\\
Z^{(2)}=Z(\epsilon_1-\epsilon_2,\epsilon_2,\vec{a}+\vec{k}\,\epsilon_2,m_i+r_i\,\epsilon_2,m_0+r_0\,\epsilon_2)
\end{align} 
Here note that $l_{\vec{k}}$ is independent of $Q=e^{-m_0}$, and $f_{\vec{k}}$ is some overall factor in the order of $Q^{\vec{k}\cdot\vec{k}/2}$. Expanding the equation by instanton fugacity $Q$, then at each $Q^{n}$ level the equation is written by 
\begin{align}
\delta_{n,0}=p_1^{r_0}Z^{(1)}_{n}+p_2^{r_0}Z^{(2)}_{n}-Z_{n}+\sum_{\vec{k}\neq 0}f_{\vec{k},r_0}l_{\vec{k}}\Bigg(\frac{Z^{(1)}_{\textrm{inst}}Z^{(2)}_{\textrm{inst}}}{Z_{\textrm{inst}}}\Bigg)\Bigg|_{O(Q^{n-\vec{k}\cdot\vec{k}/2})}.
\label{eq:be_n}
\end{align}
Since each $Z_{k}$ and $Z^{(1,2)}_{k}$ are independent of $r_0$, one can solve \eqref{eq:be_n} with three blowup equations with same $r_i$'s but different $r_0$'s. 

The blowup equations for instanton partition functions of pure YM theory with generic gauge group were already studied in \cite{Keller:2012da}. They are actually \eqref{eq:bueq_gen} with 
\begin{align}
\vec{r}_a=0,\qquad r_0=d-h^{\lor}/2
\end{align}
where $d=0,\cdots, h^{\lor}$. We extend these blowup equations to the theories with matters based on pure YM blowup equations. If one restrict the cases to $\vec{r}_a=0$, as we explained in the previous section, the $r_i$'s are technically required to be half intergers. Thus we look for the $r_0$'s that provides the correct instanton partition functions by solving \eqref{eq:be_n} while fixing $\vec{r}_a=0$ and $r_i=1/2$. Here are the results.

\begin{tabular}{|c|c|c|c|}
\hline
$G$ & matter & $r_0$ & $d$\\
\hline
$SU(N)_\kappa$ & $N_f\times\boldsymbol{N}$ & $d-N/2-\kappa/2$ & $0\leq d \leq N-|\kappa|-2N_f-1$\textcolor{red}{(?)}\\
\hline
$SU(6)_{3}$ & $1\times\boldsymbol{20}$ & $d-6/2-3/2+3/2$ & $1\leq d\leq 6$\\
\hline
$SO(7)$ & pure & $d-5/2$ & $0\leq d \leq 5$\\
\hline
$SO(7)$ & $1\times\textbf{8}$ & $d-5/2+1/2$ & $0\leq d\leq 4$\\
\hline
$SO(7)$ & $1\times\textbf{7}$ & $d-5/2+1\times1/2$ & $0\leq d\leq 4$\\
\hline
$SO(7)$ & $2\times\textbf{7}$ & $d-5/2+2\times1/2$ & $0\leq d\leq 3$\\
\hline
$G_2$ & pure & $d-4/2$ & $0\leq d \leq 4$\\
\hline
$G_2$ & $1\times\textbf{7}$ & $d-4/2+1/2$ & $0\leq d\leq 3$\\
\hline
$F_4$ & pure & $d-9/2$ & $0\leq d \leq 9$\\
\hline
$F_4$ & $1\times\textbf{26}$ & $d-9/2+1\times 3/2$ & $0\leq d\leq 6$\\
\hline
$F_4$ & $2\times\textbf{26}$ & $d-9/2+2\times 3/2$ & $0\leq d\leq 3$\\
\hline
\end{tabular}


They were tested by comparing the resulting instanton partition functions with the known results from \cite{Kim:2018gjo}($SO(7)$ and $G_2$) and \cite{DelZotto:2018tcj}($F_4$ with $N_{\boldsymbol{26}}=2$). They were compared numerically, putting random numbers on the fugacities.
Note that matters shift the $r_0$, each by one quarter of their Dynkin indices. It seems to differ from blowup formula for $SU(N)_\kappa+N_f$ instantons, where $r_0$ was affected only by its CS-level $\kappa$. However, one can rewrite the $r_0$ as
\begin{align}
r_0=&\,d-N/2-\left(\kappa+\frac{1}{2}N_f\right)/2+N_f/4\nonumber\\
=&\,d-N/2-\kappa_{\textrm{eff}}/2+N_f\times I_{\textrm{fund}}.
\end{align}
Since fundamental matters shifts the effective CS-level, they cancel their index contributions and consequently the $r_0$ apparently looks independent of matters. 

By above observations, we write the unity blowup equation for generic gauge groups and matter representations.
\begin{align}
Z(\epsilon_1,\epsilon_2,\vec{a},m_i,m_0)=\sum_{\vec{k}\in\vec{\alpha}^{\lor}}&\,Z(\epsilon_1,\epsilon_2-\epsilon_1,\vec{a}+\vec{k}\epsilon_1,m_i+\epsilon_1/2,m_0+r_0\epsilon_1)\nonumber\\
&\times\,Z(\epsilon_1-\epsilon_2,\vec{a}+\vec{k}\epsilon_2,m_i+\epsilon_2/2,m_0+r_0\epsilon_2)
\end{align}
with 
\begin{align}
r_0=d-h^{\lor}/2-\kappa_{\textrm{eff}}/2+N_{\boldsymbol{R}}\times I_{\boldsymbol{R}}.
\end{align}
Here $I_{\boldsymbol{R}}$ is the Dynkin index of $\boldsymbol{R}$ representation.

\paragraph{$SU(6)_3+1\times\boldsymbol{20}$}

As a non-trivial test, we consider the instanton partition function of the $SU(6)_3+\boldsymbol{20}$ whose 5-brane realization was found recently \cite{Hayashi:2019yxj}. Its web-diagram is given as \textcolor{red}{figure}.

\textcolor{red}{(Written before computing the $SU(6)_3+20$ instanton partition function.)}\\
Rather than comparing instanton partition functions directly, we consider an interesting Higgsing procedure. We consider the $SU(3)\times SU(3)\times U(1)\subset SU(6)$ where the $SU(6)$ multiplets are decomposed by
\begin{align}
A_{i\bar{j}}:\,\boldsymbol{35}\longrightarrow&\,(\boldsymbol{8},1)_0\oplus (1,\boldsymbol{8})_0\oplus (\boldsymbol{3},\bar{\boldsymbol{3}})_2\oplus(\bar{\boldsymbol{3}},\boldsymbol{3})_{-2}\oplus(1,1)_0,\nonumber\\
\Phi_{ijk}:\,\boldsymbol{20}\longrightarrow&\,(\boldsymbol{3},\bar{\boldsymbol{3}})_{-1}\oplus(\bar{\boldsymbol{3}},\boldsymbol{3})_1\oplus(1,1)_3\oplus(1,1)_{-3}.
\end{align} 
Here to fit with the web-diagram, we set $\Phi_{156}$ and $\Phi_{234}$ are $(1,1)_3$ and $(1,1)_{-3}$. Once $\Phi_{156}$ and $\Phi_{234}$ get non-zero VEVs, 

 When $a_5=-a_1-a_6$, the web-diagram factorizes to two $SU(3)_3$ whose Coulomb VEVs are $(a_1, a_5, a_6)$ and $(a_2, a_3, a_4)$. In the gauge theory, it can be seen partly from prepotential. The prepotential of $S(6)_3+1\times\boldsymbol{20}$ is
\begin{align}
\mathcal{F}=\frac{1}{2}m_0\sum_{i=1}^{6}a_i^2+\frac{1}{2}\sum_{i=1}^{6}a_i^3+\frac{1}{6}\sum_{i<j}(a_i-a_j)^3-\frac{1}{6}\sum_{1<i<j}(a_1+a_j+a_k)^3
\end{align}
at the Weyl chamber $a_1>\cdots>a_6$. As one sets the Coulomb VEV $a_6=-a_1-a_5$ and $a_4=-a_2-a_3$, one can check
\begin{align}
\mathcal{F}(m_0,a_1,a_2,a_3,a_4,a_5,a_6)=\mathcal{F}_{SU(3)_3}(m_0,a_1,a_5,a_6)+\mathcal{F}_{SU(3)_3}(m_0,a_2,a_3,a_4)
\end{align}
where
\begin{align}
\mathcal{F}_{SU(3)_3}(m_0,a_1,a_2,a_3)=\frac{1}{2}m_0\sum_{i=1}^{3}a_i^2+\frac{1}{2}\sum_{i=1}^{3}a_i^3+\frac{1}{6}\sum_{i<j}(a_i-a_j)^3.
\end{align}
It is Higgsed by 

%%%%%%%%%%%%%%%%%%%%%%%%%%%%%%%%%%%%%%%%%%%%%%%%%%%%%%%%%%%%%%%%

\section{Conclusion} \label{sec:conclusion}


\acknowledgments
This work is supported in part by the UESTC Research Grant A03017023801317 (SSK), the National Research Foundation of Korea (NRF) Grants 2017R1D1A1B06034369 (KL, JS), and 2018R1A2B6004914 (KHL)


\bibliographystyle{JHEP}
\bibliography{ref}

\end{document}

