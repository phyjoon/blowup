

\documentclass[letterpaper, 11pt]{article}

\usepackage{jheppub}
\usepackage{bm}
\usepackage{graphicx}
\usepackage{epstopdf}
\usepackage{amsmath, amssymb}

\usepackage{amsfonts,amssymb,epsfig,amsmath,mathtools,tabu}
\usepackage{verbatim,booktabs}

%\usepackage[utf8]{inputenc}
%\usepackage{textcomp,setspace}
\usepackage[inline]{enumitem}

\usepackage{hyperref,caption,subcaption}
\usepackage{xcolor,tikz,graphicx,afterpage}
\usetikzlibrary{shapes.geometric,positioning}
%\usetikzlibrary{cd}


\newcommand{\be}{\begin{eqnarray}}
\newcommand{\ee}{\end{eqnarray}}
\newcommand{\nn}{\nonumber}
\newcommand{\bn}{\begin{enumerate}}
\newcommand{\en}{\end{enumerate}}

%%%%%%%%%%%%%%%%%%% Figures %%%%%%%%%%%%%%%%%%%%%%%%%%%

\newcommand{\fig}[3]{
\begin{figure}
\centerline{\epsfxsize=#1\epsfbox{#2.eps}}
\newcaption{#3. \label{#2}}
\end{figure}
}

%%%%%%%%%%%%% Double line letters using amssymb %%%%%%%%%%%%%%%%

\def\identity{{\rlap{1} \hskip 1.6pt \hbox{1}}}
\def\iden{\identity}

\def\IB{\mathbb{B}}
\def\IC{\mathbb{C}}
\def\ID{\mathbb{D}}
\def\IH{\mathbb{H}}
\def\IM{\mathbb{M}}
\def\IN{\mathbb{N}}
\def\IP{\mathbb{P}}
\def\IR{\mathbb{R}}
\def\IZ{\mathbb{Z}}

%%%%%%%%%%%%%%%% Caligraphic letters %%%%%%%%%%%%%%%%%%

\def\CA{{\cal A}}
\def\CB{{\cal B}}
\def\CC{{\cal C}}
\def\CD{{\cal D}}
\def\CE{{\cal E}}
\def\CF{{\cal F}}
\def\CG{{\cal G}}
\def\CH{{\cal H}}
\def\CI{{\cal I}}
\def\CJ{{\cal J}}
\def\CK{{\cal K}}
\def\CL{{\cal L}}
\def\CM{{\cal M}}
\def\CN{{\cal N}}
\def\CO{{\cal O}}
\def\CP{{\cal P}}
\def\CQ{{\cal Q}}
\def\CR{{\cal R}}
\def\CS{{\cal S}}
\def\CT{{\cal T}}
\def\CU{{\cal U}}
\def\CV{{\cal V}}
\def\CW{{\cal W}}
\def\CX{{\cal X}}
\def\CY{{\cal Y}}
\def\CZ{{\cal Z}}

%%%%%%%%%%%%%%%%%% Greek letters %%%%%%%%%%%%%%%%%%%%%%%%%%%%

\def\a{\alpha}
\def\b{\beta}
\def\g{\gamma}
\def\d{\delta}
\def\e{\epsilon}
\def\ve{\varepsilon}
\def\z{\zeta}
% eta
\def\th{\theta}
\def\vth{\vartheta}
\def\i{\iota}
\def\k{\kappa}
\def\l{\lambda}
\def\m{\mu}
\def\n{\nu}
% xi
% o
% pi
\def\vp{\varpi}
\def\r{\rho}
\def\vr{\varrho}
\def\s{\sigma}
\def\vs{\varsigma}
\def\t{\tau}
\def\u{\upsilon}
% phi
\def\vph{\varphi}
% chi
\def\ch{\chi}
% psi
\def\w{\omega}
%
\def\G{\Gamma}
\def\D{\Delta}
\def\Th{\Theta}
\def\L{\Lambda}
% Xi
% Pi
\def\S{\Sigma}
\def\Y{\Upsilon}
% Phi
% Psi
\def\O{\Omega}


%%%%%%%%%%%%%%%%% Mathematical Symbols %%%%%%%%%%%%%%%%%%%%%%%%%%%%

\def\half{\frac{1}{2}}
\def\thalf{{\textstyle \frac{1}{2}}}
\def\imp{\Longrightarrow}
\def\goto{\rightarrow}
\def\para{\parallel}
\def\vev#1{\langle #1 \rangle}
\def\del{\nabla}
\def\grad{\nabla}
\def\curl{\nabla\times}
\def\div{\nabla\cdot}
\def\p{\partial}
\newcommand{\bra}[1]{\langle{#1}|}
\newcommand{\ket}[1]{|{#1}\rangle}
\def\fslash{\displaystyle{\not}}

%%%%%%%%%%%%%%%%%%%% Normal font in math %%%%%%%%%%%%%%%%%%%%%%%%%%

\def\Tr{{\rm Tr}}
\def\tr{{\rm tr}}
\def\det{{\rm det}}



%%%%%%%%%%%%%%%%%%%%%%%%%%%%%%%%%%%%%%%%%%%%%%%%%%%%%
\title{Instantons from Blow-up}

\author[a]{Joonho Kim,}
\author[b]{Sung-Soo Kim,}
\author[c]{Ki-Hong Lee,}
\author[a]{Kimyeong Lee,}
\author[a]{and Jaewon Song}
\affiliation[a]{School of Physics, Korea Institute for Advanced Study, Seoul 02455, Korea}
\affiliation[b]{School of Physics, University of Electronic Science and Technology of China,\\ No.4, Section 2, North Jianshe Road, Chengdu, Sichuan 610054, China}
\affiliation[c]{Department of Physics and Astronomy \& Center for Theoretical Physics\\ Seoul National University, Seoul 08826, Korea}
\emailAdd{joonhokim@kias.re.kr}
\emailAdd{sungsoo.kim@uestc.edu.cn}
\emailAdd{khlee11812@gmail.com}
\emailAdd{klee@kias.re.kr}
\emailAdd{jsong@kias.re.kr}

\abstract{
The Nekrasov partition function for 4d $\CN=2$ or 5d $\CN=1$ gauge theory on the blow up of a point $\hat{\IC}^2$ can be written in terms of the partition function on the flat space $\IC^2$. At the same time, the partition function on the blow up is identical to the partition function on a flat space for sufficiently large class of examples. 
This relation enables us to compute the instanton partition functions for 4d $\CN=2$ and 5d $\CN=1$ gauge theories for arbitrary gauge theory with large class of matter representations without knowing explicit construction of the instanton moduli space. Remarkably, the instanton partition function is completely determined by the perturbative part. 
We obtain the partition function for the previously unknown theories: exceptional gauge groups $EFG$ with fundamental/spinor hypermultiplets and more. We also compute the case with $SU(6)$ with rank-3 antisymmetric tensor and compare with the topological vertex computation using the recently found 5-brane web configuration. 
}


\preprint{KIAS-P19???, SNUTP19-???}

%%%%%%%%%%%%%%%%%%%%%%%%%%%%%%%%%%%%%%%%%%%%%%%%%%%
\begin{document}
\maketitle

\section{Introduction} \label{sec:intro}


The Seiberg-Witten prepotential provides a complete description for the low energy dynamics of 4d $\mathcal{N}=2$ or 5d $\mathcal{N}=1$ gauge theory in its Coulomb branch \cite{Seiberg:1994rs,Seiberg:1994aj}. It is a function of the vacuum expectation value (VEV) of the scalar in the vector multiplet that parameterizes the Coulomb branch moduli space. Quantum correction to the prepotential is known to be one-loop exact, while there also exist non-perturbative corrections coming from Yang-Mills instantons. 

An efficient way to compute the fully quantum corrected prepotential $\mathcal{F}$ is to study the Nekrasov partition function $\mathcal{Z}$ on $\Omega$-deformed $\mathbb{C}^2$ or $\mathbb{C}^2 \times S^1$. 
It can be written as the product of the classical, one-loop, and instanton contributions,
\begin{align}
  \mathcal{Z}(\vec{a}, \vec{\mathfrak{m}}, \e_1, \e_2, \mathfrak{q}) = \mathcal{Z}_{\textrm{class}}(\vec{a}, \e_1, \e_2, \mathfrak{q}) \ \mathcal{Z}_{\textrm{1-loop}} (\vec{a}, \vec{\mathfrak{m}}, \e_1, \e_2) \ \mathcal{Z}_{\textrm{inst}}(\vec{a}, \vec{\mathfrak{m}}, \e_1, \e_2, \mathfrak{q}),
\end{align}
where the instanton piece is the fugacity sum over all multi-instanton contributions:
\begin{align}
  \mathcal{Z}_{\textrm{inst}}(\vec{a}, \vec{\mathfrak{m}}, \e_1, \e_2, \mathfrak{q}) = 1 + \sum_{k=1}^\infty \mathfrak{q}^k \mathcal{Z}_k(\vec{a}, \vec{\mathfrak{m}}, \e_1, \e_2).
\end{align}
Once the Nekrasov partition function is known, one can extract the prepotential via taking $\e_i \to 0$ limit as $\mathcal{F} = \lim_{\epsilon_{i}\rightarrow 0} \epsilon_1 \epsilon_2 \log{\mathcal{Z}}$ \cite{Nekrasov:2002qd,Nekrasov:2003rj,Nakajima:2003pg, Braverman:2004cr}.

The instanton part of the partition function in the $\Omega$-background can be computed once we know appropriate instanton moduli space. For the classical gauge group, the ADHM construction of the moduli space provides a direct way compute the instanton partition function. The ADHM construction can be understood as the quantum mechanics described by the Dp-D(p+4) system. The Higgs branch moduli space of the Dp system gives the desired moduli space. Matter fields can be also introduced by considering the world-volume theory on the D0-branes of the D0-D4-D8 system. 
By using the localization on the 1d system on the D0-branes or its dimensional reduction \cite{Moore:1997dj, Bruzzo:2002xf}, the partition function has been obtained for variety of cases: classical gauge groups \cite{Nekrasov:2004vw, Marino:2004cn, Fucito:2004gi, Hollands:2010xa, Hollands:2011zc}, exceptional gauge groups \cite{Benvenuti:2010pq, Keller:2011ek, Keller:2012da, Hanany:2012dm}. 
3d N=4 Coulomb branch realization of the instanton moduli space: \cite{Cremonesi:2014xha}
The precise choice of the Contour of the ADHM integral has been derived in \cite{Hwang:2014uwa, Hori:2014tda, Cordova:2014oxa} following the Jeffrey-Kirwan residue formula in 2d elliptic genus \cite{Benini:2013xpa,Benini:2013nda}. 

But there is no ADHM type construction for the exceptional gauge groups or generic type of matter fields. String-theoretic picture implies that they require strong-coupling dynamics or non-Lagrangian field theories to realize instanton moduli space of exceptional group as a vacuum moduli space. There has been a few results regarding the exceptional instantons. 

In this paper, we generalize the approach of Nakajima-Yoshioka (NY) \cite{Nakajima:2003pg,Nakajima:2003uh,Nakajima:2005fg} for the pure YM theory. In \cite{Keller:2012da}, the NY blow-up formula were used to compute the instanton partition function for exceptional gauge group and tested against the superconformal index of 4d SCFT where the Higgs branch is given by the instanton moduli space \cite{Gaiotto:2012uq}. We propose a general blow-up formula for general gauge theory with arbitrary representations, under the condition that the matter representation is not too large.\footnote{This has to do with the 1-loop beta function coefficient  as we will discuss in detail.} This enables us to compute the instanton partition functions for numerous gauge theories that have not been known before, without relying on to the explicit construction of the moduli space. 

The basic idea is as follows: Let us consider a one-point blow up $\hat{\IC^2}$ of the flat space $\IC^2$. The \emph{full} partition function on $\hat{\IC^2}$ can be written in terms of the products of the \emph{full} partition function of $\IC^2$. But at the same time, the partition function on the blowup is identical to that of the flat space since we can smoothly blow-down $\hat{\IC^2}$ to $\IC^2$ as long as the matter representation is not `too large'. This provides us a functional relation for the partition function, which turns out to be sufficient to determine the instanton partition function itself. Remarkably, this relation is completely determined by the perturbative part of the partition function. Therefore we arrive at a surprising conclusion: The \emph{perturbative} physics determine the \emph{non-perturbative} physics! 

For example, we find the following universal expression for the 1-instanton partition function with arbitrary gauge group and matters:
\begin{align}
Z_1 &= \frac{e^{\e_1+\e_2} }{(1-e^{\e_1})(1-e^{\e_2})} \sum_{\g \in \Delta_l} \frac{e^{(h^\vee -1)a_\g/2} \prod_{\g \cdot w = 1} (1-e^{a_w + m^{\textrm{phy} }}) }{ (e^{a_\g/2} - e^{-a_\g/2})(1-e^{a_\g-\e_1-e_2}) {\prod_{\g \cdot \a = 1} (e^{a_\a/2}-e^{-a_\a/2 })} } \ , 
\end{align}
where ...


\paragraph{Notation}
The equivariant parameters $\epsilon_{1}, \epsilon_2, \vec{a}$ are associated to the $U(1)^2 \times U(1)^{|G|}$ action on the $k$-instanton moduli space $\mathcal{M}_{k,G}$. Additional parameters $\vec{\mathfrak{m}}$ are introduced if the gauge theory has an extra  flavor symmetry. The instanton fugacity $\mathfrak{q}$ can be written as $\mathfrak{q}\equiv\Lambda^{b_0}$ (4d) or $\mathfrak{q}\equiv \exp(2\pi i \Lambda)$ (5d) where $\Lambda$ denotes the bare coupling of the 4d/5d gauge theory.

Throughout Section~\ref{sec:blowup}, we will characterize the partition function in terms of $m_i \equiv \mathfrak{m}_i + \epsilon_+$ (where $\epsilon_+ \equiv \frac{\epsilon_1 + \epsilon_2}{2}$) as a reflection of the Donaldson-Witten twist by $SU(2)_R$ symmetry. Here $m_i$ corresponds to the `physical' mass parameter of a matter hypermultiplet. We will also stick to the `effective' instanton fugacity $q \equiv \mathfrak{q} e^{{b_0}\epsilon_+}$, which absorbs the $SU(2)_R$ generated mass of fermionic instanton zero modes, in describing the 5d Nekrasov partition function.

%%%%%%%%%%%%%%%%%%%%%%%%%%%%%%%%%%%%%%%%%%%%%%%%%%%%%%%%%%%%%%%%

\section{Instanton Counting from Blow-up} \label{sec:blowup}



The essential idea of using the blow-up of ${\IC}^2$ for instanton counting is that the gauge theory partition function for a 4d $\CN=2$ (or 5d $\CN=1$) theory on the blow-up of a point $\hat{\IC}^2$ (or $S^1 \times \hat{\IC}^2$) can be written in two different ways. This will allow us to write a recursion relation for the instanton partiton function that can be solved rather easily \cite{Nakajima:2003pg, Nakajima:2003uh,Nakajima:2005fg, Keller:2012da}. 






\paragraph{Localization on the blow-up $\hat{\IC}^2$}

One of the expressions for the partition function $\hat{\mathcal{Z}}$ on the blow-up $\hat{\IC}^2$ comes from the Coulomb branch localization, which results that $\hat{\mathcal{Z}}$ can be obtained by patching together the flat-space partition function $\mathcal Z$ \cite{Nekrasov:2003vi}.

The blow-up $\hat{\IC}^2$ is constructed from $\IC^2$ by replacing the origin with the compact 2-cycle. In particular, the geometry near the $\IP^1$ is identical to a line bundle of degree $(-1)$ on the $\IP^1$. One can parametrize $\mathcal{O}(-1)\rightarrow \IP^1$ using three homogeneous coordinates $(z_0, z_1, z_2)$, satisfying the projective condition $(z_0, z_1, z_2) \sim (\lambda^{-1}z_0, \lambda^1 z_1, \lambda^1 z_2)$ for any $\lambda \in \IC^*$, where the two-cycle  $\IP^1 \subset \hat{\IC}^2$ corresponds to the locus $z_0 = 0$. 
%  can be described as a subspace of $\IC^2 \times \IP^1$, defined as
% \begin{align}
%  \left\{ \big((x, y), [z: w]\big) \in \IC^2 \times \IP^1 \,\big|\, xw = yz \,\right\} \ , 
% \end{align}
% where $[z:w]$ represents the homogeneous coordinates on $\IP^1$. Notice that this space 
We are interested in the $U(1)^2$ equivariant partition function, with the $U(1)^2$ action $V$ rotating the complex coordinates $(z_0, z_1, z_2)$ as follows:
\begin{align}
  (z_0, z_1, z_2) \mapsto (z_0, e^{\e_1}z_1, e^{\e_2}z_2).
%  \big((x, y), [z: w]\big) \mapsto \big((e^{ \e_1} x, e^{ \e_2} y), [e^{ \e_1} z: e^{ \e_2} w]\big) \ . 
\end{align}
Instantons are located at two fixed points of the $U(1)^2$ action, i.e., the north/south poles of the $\mathbb{P}^1$, whose coordinates are 
% $((x, y), [z: w]) = ((0, 0), [1, 0])$ and $((0, 0), [0, 1])$ 
$(z_0, z_1, z_2)  = (0,1,0)$ and $(0,0,1)$. Around these fixed points, the local weights under the $U(1)^2$ action $V$ are:
\begin{align}
  (z_0 z_1,\,z_2/z_1) &\mapsto  (e^{\e_1}z_0 z_1, \,e^{\e_2 - \e_1}z_2/z_1) &  \text{(near the north pole)}\\
  (z_0 z_2,\,z_1/z_2) &\mapsto (e^{\e_2}z_0 z_2, \,e^{\e_1 - \e_2}z_1/z_2) &  \text{(near the south pole)}
\end{align}

The full partition function $\hat{\mathcal Z}$ on $\hat{\IC}^2$, which includes both the perturbative and instanton contributions, can be obtained by performing the localization on the Coulomb branch. On the Coulomb branch, the gauge group is generically broken to $U(1)^r$ where $r$ is the rank of the gauge group. The $U(1)^r$ equivariant parameters $\vec{a}$ naturally appear in the partition function. One needs to sum over all distinct field configurations with zero-sized instantons located at the north and south poles. All the inequivalent configurations are labeled by the vector $\vec{k}$ of the first Chern numbers, corresponding to different flux configurations on the two-cycle $\mathbb{P}^1$. Here we  turn off any possible external flux that can be supported on the $\IP^1$. 
Summing up, $\hat{\mathcal Z}$ can be expressed as the following formula \cite{Nekrasov:2003vi, Gottsche:2006bm, Gottsche:2006tn, Gasparim:2008ri, Bonelli:2012ny}:
\begin{align} \label{eq:blowup}
  \hat{\mathcal Z}(\vec{a}, \e_1, \e_2) = \sum_{\vec{k} \in \Lambda} {\mathcal Z}(\vec{a}+ \vec{k} \e_1, \e_1, \e_2 - \e_1) {\mathcal Z}(\vec{a}+\vec{k} \e_2, \e_1 - \e_2, \e_2) 
 \end{align}
 where the flux sum is taken over the coweight lattice $\Lambda$  of the gauge algebra. Notice that the Coulomb parameter $\vec{a}$ gets an appropriate shift at each fixed point $p$, induced by the non-trivial magnetic flux $\vec{k}$ on the blown-up $\IP^1$, with the proportionality constant $H|_p$. Values of the moment map $H$ for the $U(1)^2$ action $V$, i.e., $dH = \iota_V \omega$, at the north and south poles are 
\begin{align}
  H|_\text{NP} = \epsilon_1 \text{ and } H|_\text{SP} = \epsilon_2.
\end{align}


\paragraph{Partition function on $\hat{\IC}^2$ vs $\IC^2$}
An alternative fact for the partition function $\hat{\CZ}$ on the blow-up $\hat{\IC}^2$ is that $\hat{\CZ}$ is actually identical to the flat-space partition function $\CZ$  \cite{Nakajima:2003pg, Nakajima:2003uh,Nakajima:2005fg}.
 


% This can be understood as follows: 
The blow-up $\hat{\IC}^2$ is identical to $\IC^2$ except for the origin, which is replaced by the blown-up sphere $\IP^1$. 
Since the Nekrasov partition function gets contributions only from the small instantons localized at the fixed points of the $U(1)^2$ equivariant action $V$, the size of the divisor should not affect the partition function as we smoothly shrink it. So we expect that $\hat{\CZ} = \CZ$. This implies the following  relation: \cite{Nakajima:2003pg, Nakajima:2003uh,Nakajima:2005fg,Keller:2012da,Gu:2018gmy,Gu:2019dan}
\begin{align} 
  \label{eq:blowup-generic}
  \CZ(\vec{a}, \e_1, \e_2) = \sum_{\vec{k} \in \Lambda} \CZ(\vec{a}+ \vec{k} \e_1, \e_1, \e_2 - \e_1) \CZ(\vec{a}+\vec{k} \e_2, \e_1 - \e_2, \e_2) \ , 
\end{align}

The blow-up identity can also be generalized to orbifold partition functions \cite{Sasaki:2006vq,Bonelli:2012ny,Ito:2013kpa}. For example, if we consider the total space of the bundle $\CO(-2) \to \IP^1$ and shrink the base, we land on the orbifold $\IC^2/\IZ_2$ with singularity at the origin. There are certain states originated from wrapped D-branes on the $\IP^1$ which become massless by shrinking the $\IP^1$. They constitute to the required twisted sector of the orbifold partition function. In any case, we expect that the Nekrasov partition function still remains the same by blowing up and down.\footnote{This simple picture does not necessarily hold when there are too many hypermultiplets, due to some subtle scheme dependence related to wall-crossing \cite{Gottsche:2006bm,Gottsche:2006tn, Ito:2013kpa}. We will discuss about this issue in Section~\ref{}.}



\paragraph{Background parameters}

The expected relation \eqref{eq:blowup-generic} is only a schematic expression.
The partition function also depends on some background parameters, such as the gauge coupling  $\Lambda$ and flavor chemical potentials $m$. They need to be appropriately shifted at each fixed point $p$ of the blow-up $\hat{\IC}^2$, keeping invariant the effective mass parameters twisted by $SU(2)_R$.

One can identify the shifted parameters $\Lambda|_p$ and $m|_p$ at a fixed point $p$, by examining the 1-loop effective free energy of the 5d Nekrasov partition function $\CZ$. For a general 5d $\mathcal{N}=1$ gauge theory with the Chern-Simons level $\kappa$ and/or some hypermultiplets,
\footnote{We assume a particular Weyl chamber in the Coulomb branch, i.e., $ 0< a_i < \e_+ < m$ for all $i\in \{1,\cdots, r\}$.}
\begin{align}
  \label{eq:free-energy}
  & \log{\CZ} = \frac{1}{\epsilon_1\epsilon_2}\Bigg[\frac{1}{2}\Lambda \, h_{ij}a_i a_j +\frac{\kappa}{6}d_{ijk} a^{i}a^j a^k +  \sum_{\vec{\alpha}\in\Delta}\left(\frac{(\vec{a}\cdot\vec{\alpha}+\e_+)^3}{12}-\frac{\e_1^2+\e_2^2+24}{48}\,(\vec{a}\cdot\vec{\alpha}+\e_+)+1\right)\nn \\
  &-\sum_i\sum_{\vec{\omega}\in\boldsymbol{R}_i}\left(\frac{(\vec{a}\cdot\vec{\omega}+m_i)^3}{12}-\frac{\e_1^2+\e_2^2+24}{48}\,(\vec{a}\cdot\vec{\omega}+m_i)+1\right)\Bigg] + \sum_{n=1}^\infty\frac{ f(n\vec{a},n\epsilon_{1,2},nm_i)}{n}
\end{align}
in which the first 2 terms inside the square brackets constitute the classical prepotential $\CF_\text{cl}$, whereas all the other terms come in at one-loop level.  The set of all roots is denoted by $\Delta$. The $\vec{\omega}$ runs over all weight vectors in the representation $\mathbf{R}_i$ of the $i$-th hypermultiplet. First, let us consider a 5d free hypermultiplet with mass $m$, whose corresponding free energy is 
\begin{align}
  \label{eq:free-hyper}
  & \log{\CZ} = -\frac{1}{\epsilon_1\epsilon_2} \left(\frac{m^3}{12}-\frac{\e_1^2+\e_2^2+24}{48}\,m +1\right) + \sum_{n=1}^\infty\frac{ 1}{n} \frac{e^{-nm}\cdot e^{-n\e_+}}{(1-e^{-n\e_1})(1-e^{-n\e_2})}.
\end{align}
This satisfies the relation \eqref{eq:blowup-generic} only when it is accompanied by $m \rightarrow m \pm \frac{H}{2}|_p$. 
Each $\pm$ shift preserves the combination $(m \pm \e_+)$ respectively,  corresponding to the $SU(2)_R$ twisted mass of the hypermultiplet \cite{Okuda:2010ke}. Throughout this paper, we take $m_\text{phy} \equiv (m- \e_+)$ as the invariant $SU(2)_R$ twisted mass, identifying that
\begin{align}
  \label{eq:mass-shift}
  m|_p = m - \frac{H}{2}|_p
\end{align}
at a fixed point $p$. Second, the twisted instanton mass can be easily read off after simplifying \eqref{eq:free-energy} based on the following identities:
\begin{align}
  \textstyle \sum_{\vec{\omega}\in\boldsymbol{R}}(\vec{a}\cdot\vec{\omega})(\vec{b}\cdot\vec{\omega})(\vec{c}\cdot\vec{\omega})=&\,I_3(\boldsymbol{R})\,d_{ijk}\,a^ib^jc^k,\nonumber \\
  \textstyle \sum_{\vec{\omega}\in\boldsymbol{R}}(\vec{a}\cdot\vec{\omega})(\vec{b}\cdot\vec{\omega})=&\,I_2(\boldsymbol{R})\,h_{ij}\,a^ib^j,\\
  \textstyle \sum_{\vec{\omega}\in\boldsymbol{R}}(\vec{a}\cdot\vec{\omega})=&\,0.\nonumber
\end{align}
Here $I_2(\mathbf{R})$ and $I_3(\mathbf{R})$ are the quadratic and cubic Dynkin indices\footnote{The Dynkin indices are normalized such that $I_2(\mathbf{F}) = 1$ for the $SU(N)$ fundamental representation $\bf F$.} of a given representation $\bf R$. In particular, $I_2(\mathbf{adj}) = h^\vee$ is the dual Coxeter number of the gauge algebra.
Speaking explicitly, the rational part of the one-loop free energy  \eqref{eq:free-energy} can be written as 
\begin{align}
  \log{\CZ}=\frac{1}{\e_1\e_2}\Bigg[&\,\frac{1}{2}\Lambda_\text{eff}\,h_{ij}\,a^ia^j+\frac{\k_\text{eff}}{6}d_{ijk}\,a^ia^ja^k+ \text{(independent of $\vec{a}$)}\Bigg],
  %  + \sum_{n=1}^\infty\frac{ f(n\vec{a},n\epsilon_{1,2},nm_i)}{n}
  % + \text{(exponential part)}
  \end{align}
where the one-loop corrected parameters are 
\begin{align}
  \Lambda_{\textrm{eff}}=\,\Lambda+\Bigg(h^{\vee}-\sum_i\frac{I_2(\boldsymbol{R}_i)}{2}\Bigg)\e_+-\sum_i\frac{I_2(\boldsymbol{R}_i)}{2}m_i^{\textrm{phy}},\quad 
  \k_{\textrm{eff}}=\,\k-\sum_i\frac{I_3(\boldsymbol{R}_i)}{2}
\end{align}
in which the sum is taken over all the matter multiplets. We notice that the instanton soliton carries the $SU(2)_R$ twisted effective mass, given by $m_\text{inst} \equiv \Lambda_{\textrm{eff}} + \kappa_\text{eff}\,\e_+$.
In addition, the  effective Chern-Simons coupling $\k_{\textrm{eff}}$ induces an electric charge to the instanton, contributing to its ground state energy as $E_0 = m_\text{inst} + \vec{a} \cdot \vec{\Pi}$, where $\vec{\Pi}$ is the $U(1)^r \subset G$ electric charge.\footnote{This agrees with the supersymmetric Casimir energy of the ADHM quantum mechanics.} To keep it invariant the effective instanton mass $m_\text{inst}$ at a fixed point $p$ of the blow-up $\hat{\IC}^2$, we require the shifted gauge coupling $\Lambda|_p$  to be 
\begin{align}
  \label{eq:coupling-shift}
  \Lambda|_p = \Lambda + \frac{b}{2}H|_p \quad \text{ with }\quad
  b \equiv h^{\vee}-\sum_i\frac{I_2(\boldsymbol{R}_i)}{2} + \kappa_\text{eff}.
\end{align}
For example, the $\IC^2 \times S^1$ partition function of $\mathcal{N}=1$ free Maxwell theory, given as a product between the perturbative index of a free vector multiplet,
\begin{align}
  % \CZ_\text{U(1)}^{\rm pert}(\epsilon_1, \epsilon_2) &= 
    \exp\left[+\frac{1}{\e_1 \e_2} \left(\frac{\epsilon _1
  \epsilon _2 (\epsilon _1+\epsilon _2)}{48} -\frac{\epsilon _1+\epsilon _2}{4} +1 \right)  \right] \cdot  \text{PE}\left[-\frac{e^{-\e_1-\e_2}}{(1-e^{-\e_1})(1-e^{-\e_2})}\right],
\end{align}
and the non-perturbative correction from $U(1)$ multi-instantons,
\begin{align}
  % \CZ_\text{U(1)}^{\rm pert}(\epsilon_1, \epsilon_2) &= 
  \text{PE}\left[\frac{e^{-\Lambda} \cdot e^{-\e_+}}{(1-e^{-\e_1})(1-e^{-\e_2})}\right],
\end{align}
will comply with the relation \eqref{eq:blowup-generic} by shifting the gauge coupling as $\Lambda \rightarrow \Lambda|_p = \Lambda + \frac{H}{2}|_p$. 

In summary, the blow-up  relation \eqref{eq:blowup-generic} should always be understood with \eqref{eq:mass-shift} and \eqref{eq:coupling-shift}.



\paragraph{Correlation functions}

In fact, \eqref{eq:blowup-generic} is not enough to fix the partition function completely,  since there are 3 unknown functions and only one relation. More independent relations would be found from insertion of non-trivial $\mathcal{Q}$-closed operators \cite{Nakajima:2003pg,Nakajima:2005fg}. So we consider the correlation functions of that operator $\mathcal{O}_{\IP^1}$ associated to the two-cycle $\IP^1$ of the blow-up $\hat{\mathbb{C}}^2$. 

In the 4d Donaldson-twisted theory, the $\mathcal{Q}$-invariant observable $\mathcal{O}_{\IP^1}$ can be constructed by applying the descent procedure twice to the Casimir invariant $\text{Tr}(\Phi^2)$. It can be written in terms of the component fields as follows: \cite{Witten:1988ze}
\begin{align} \label{eq:muC}
  {\CO}_{\IP^1} = \int \left\{ \omega \wedge \text{Tr}\Big(\Phi F + \half \psi \wedge \psi\Big) - H\, \text{Tr}\Big( F \wedge F \Big) \right\}  \ .
\end{align} 
We find it convenient to study the generating function $\langle e^{t \, {\CO}_{\IP^1}}\rangle $ of the correlators $\langle {\CO}_{\IP^1} \ldots {\CO}_{\IP^1} \rangle$. 
In particular, the insertion of $\exp{(t  {\CO}_{\IP^1})}$ causes an extra shift of the instanton mass by $2tH|_p$ at a fixed point $p$ of the blow-up $\hat{\IC}^2$ \cite{Nakajima:2003pg,Nakajima:2003uh, Nakajima:2005fg}. This modifies \eqref{eq:coupling-shift} to $\Lambda|_p = \Lambda + (\frac{b}{2}+t)H|_p$.
Therefore the expectation value of the generating function is given by
\begin{align}
  \label{eq:corgen}
  \langle e^{t \, \mu(C)}\rangle = \sum_{\vec{k} \in \Lambda}  \frac{\CZ^{(N),t}(\vec{k}) \cdot \CZ^{(S),t}(\vec{k})}{\CZ^{(0)}}\ , 
\end{align} 
where
\begin{align}
\begin{split}
  \CZ^{(N),t}(\vec{k}) &\equiv \CZ(\vec{a}+\vec{k} \e_1, \e_1, \e_2-\e_1, \Lambda +  (\tfrac{b}{2}+t)\e_1, \vec{m} +\tfrac{1}{2}\e_1) \ , \\
  \CZ^{(S),t}(\vec{k}) &\equiv \CZ(\vec{a}+\vec{k} \e_2, \e_1 - \e_2, \e_2, \Lambda +  (\tfrac{b}{2}+t)\e_2, \vec{m} +\tfrac{1}{2}\e_2)\ , \\
 \CZ^{(0)} &\equiv \CZ(\vec{a}, \e_1, \e_2, \Lambda, \vec{m}).
\end{split}
\end{align}
As we shrink the two-cycle $\IP^1$ to recover the flat $\IC^2$, the effect of  inserting ${\CO}_{\IP^1}^d$ turns out to give a vanishing contribution due to the selection rule. We recall that the instanton breaks the $U(1)_R$ symmetry 
to the a discrete subgroup \textcolor{red}{$\IZ_{2h^\vee - \sum_i I_2(\mathbf{R}_i)}$} under which the operator ${\CO}_{\IP^1}^d$ carries a charge of $+2$. So the correlation functions vanish unless the discrete charges add up to zero, modulo $4h^\vee - 2\sum_i I_2(\mathbf{R}_i)$. Therefore, expanding \eqref{eq:corgen} in powers of $t$, we find
\begin{align}
  \label{eq:donaldson-4d}
  \langle e^{t \, \mu(C)}\rangle = 1 + \CO(t^{2h^\vee - \sum_i I_2(\mathbf{R}_i)})  \ . 
\end{align}
As long as the hypermultiplet representation is not too large, i.e., when $2h^\vee - \sum_i I_2(\mathbf{R}_i) > 2$, this allows us to write 3 independent relations for the 3 unknown varables. One can expand $\langle e^{t \, \mu(C)}\rangle$ until $\mathcal{O}(t^2)$ order and recursively solve for $\CZ^{(0)}$ at every instanton number. So the instanton part of the partition function will be completely determined from the perturbative partition function. An explicit form of the recursion relation will be displayed in Section~\ref{subsec:recursion}.

Turning to the 5d $\CN=1$ gauge theory on $S^1$, the $\mathcal{Q}^2$ now involves translation along the circle. The Casimir operator $\Tr (\Phi^2)$ and its descendants are no longer well-defined operators, but instead we need to 


the operator $\mu(C)$ is naturally uplifted via exponentiation as follows: \cite{Baulieu:1997nj}
\begin{align} 
  \mu(C)^d = \exp\left[
    d\int \left\{ \omega \wedge \text{Tr}\Big(A \wedge dA + \frac{2}{3}A^3 \Big)  + \omega \wedge \Big( \phi \, F + \frac{1}{2}\psi \wedge \psi \Big)\wedge dt + \frac{H}{2}\, \text{Tr}\Big( F \wedge F \Big) \right\}
     \right] \ .
\end{align}
This stems from the fact that $\Tr (\Phi^2)$ is . Instead, to take into account the periodicity, we need to 


exponentiate to obtain the `Coulomb parameter'. The $\mu(C)$ operator has to be exponentiated accordingly to take into account the periodicity and the coefficient in front of the exponent has to be properly quantized. More precisely, the offset $d_0$ is set to be $d_0 \equiv \frac{h^\vee}{2} - \sum_{i} \frac{I_2(\mathbf{R_i})}{4}$ where the sum is taken over all matter multiplets, since \textcolor{red}{(explain the reason)}. Then $d$ is integer-valued.  


% $\langle \mu(C) \ldots \mu(C)  \rangle$ of operators 
Here $\omega$ and $H$ are the K\"ahler two-form on $C$ and the moment map \eqref{eq:moment-map} associated to $V$.



5d operator

coefficient quantization 

unity condition (?)

violation? wall crossing etc.



\pagebreak





in 5d SYM on $S^1$





we no longer have a discrete R-symmetry to impose the selection rule. 


The natural uplift of the 4d operator 
is given via  
where the 4d complex scalar $\Phi_4$ is decomposed into
\begin{align}
  \int\omega \wedge \text{Tr}(\Phi_4 F) \ \longrightarrow \
  \int\omega \wedge \left(\text{CS}_3(A) + \text{Tr}(\Phi_5 F) \right) \in \mu(C).
\end{align}
% We can always choose our scale so that $\beta = 1$ and we will omit this parameter. 

\textcolor{red}{[TODO: Explain why $e^{\mu(C)}$ is good topological operator in 5d. Explain why inserting this does not alter the partition function.]}

Therefore, we obtain
\begin{align} \label{eq:blowup5d}
 Z = \sum_{\vec{k} \in \Lambda} \exp\left( (d+d_0)\, \mu(\vec{k}) \right) Z^{(N), d}(\vec{k})  Z^{(S), d}(\vec{k})  \qquad (d=0, 1, \ldots, h^\vee - I(R)/2 ) \ , 
\end{align} 
for the 5d theories. We call the relation above as the blow-up equation.  \textcolor{red}{(up to here)}

\subsection{Instanton partition function from the blow-up equation}
\label{subsec:recursion}
From the blow up equation \eqref{eq:blowup5d}, we can determine the instanton partition function for the 5d $\CN=1$ theory on $S^1 \times \IC^2$. To see this, let us decompose the partition function in terms of classical, 1-loop and the instanton piece:
\begin{align}
 Z(\vec{a}, m, \e_1, \e_2, q) = Z_{\textrm{class}}(\vec{a}, \e_1, \e_2, q) Z_{\textrm{1-loop}} (\vec{a}, m, \e_1, \e_2) Z_{\textrm{inst}}(\vec{a}, m, \e_1, \e_2, q)
\end{align}
Then the blow-up equation can be rewritten as
\begin{align}
 Z_{\textrm{inst}} = \sum_{\vec{k}} \left[ \exp\left( (d+d_0) \mu(\vec{k}) \right)  \frac{Z^{(N), d}_{\textrm{pert}} Z^{(S), d}_{\textrm{pert}}}{Z_{\textrm{pert}}} \right] Z^{(N), d}_{\textrm{inst}} Z^{(S), d}_{\textrm{inst}}  \ , 
\end{align}
where $Z_{\textrm{pert}} \equiv Z_{\textrm{class}} Z_{\textrm{1-loop}}$ and the superscript $(N/S)$ denotes appropriate shift of parameters in the north/south pole 
\begin{align}
\begin{split}
 Z^{(N), d}(\vec{k}) &\equiv Z(\vec{a}+\vec{k} \e_1, m, \e_1, \e_2-\e_1, q e^{(d+d_0)\e_1}) \ , \\
 Z^{(S), d}(\vec{k}) &\equiv Z(\vec{a}+\vec{k} \e_2, m, \e_1 - \e_2, \e_2, q e^{(d+d_0)\e_2}) \ . 
\end{split}
\end{align}
 Notice that there is no $q$-dependence in the 1-loop part, hence no $d$-dependence as well. 
The instanton piece can be expanded in terms of the instanton number as
\begin{align}
 Z_{\textrm{inst}} (\vec{a}, \e_1, \e_2, q) = \sum_{n \ge 0} q^n Z_n (\vec{a}, \e_1, \e_2) \ . 
\end{align}
Plugging in, we obtain
\begin{align}
 q^n Z_n = \sum_{\vec{k}} e^{(d+d_0)\left( \vec{k} \cdot \vec{a} + \half \vec{k} \cdot \vec{k} (\e_1 + \e_2)\right)} q^{\half \vec{k}\cdot \vec{k}} \left( \frac{Z^{(N)}_{\textrm{1-loop}} Z^{(S)}_{\textrm{1-loop}}}{Z_{\textrm{1-loop}}} \right) q^{\ell + m} e^{(d+d_0)(\ell \e_1 + m \e_2)} Z^{(N)}_{\ell} Z^{(S)}_m  \ , 
\end{align}
where $q^{\half \vec{k} \cdot \vec{k}}$ comes from the classical piece as we will see later. Collecting the terms with the same order, we obtain
\begin{align}
 Z_n = \sum_{\half \vec{k}\cdot\vec{k} + \ell + m = n} e^{(d+d_0)\left( \vec{k} \cdot \vec{a} + \half \vec{k} \cdot \vec{k} (\e_1 + \e_2) + \ell \e_1 + m \e_2 \right)} g(\vec{k}) Z^{(N)}_{\ell}(\vec{k}) Z^{(S)}_m (\vec{k}) \ , 
\end{align}
where $g(\vec{k}) \equiv \frac{Z^{(N)}_{\textrm{1-loop}} Z^{(S)}_{\textrm{1-loop}}}{Z_{\textrm{1-loop}}}$. Notice that the right-hand side of the above equation only involves the instanton parts with $\ell , m < n$. Therefore, one can determine the $n$-instanton partition function recursively from $Z_0 = 1$. 
To do this, let us separate $n$-instanton contribution from the above expression. We get 
\begin{align}
 Z_n = e^{\e_1 (d+d_0) n} Z^{(N)}_n + e^{\e_2 (d+d_0) n} Z^{(S)}_n + I_n^{(d)}(\vec{a}, m, \e_1, \e_2) \ , 
\end{align}
where
\begin{align}
I_n^{(d)}(\vec{a}, m, \e_1, \e_2) = \sum_{\substack{\half \vec{k}\cdot\vec{k} + \ell + m = n \\ \ell, m < n}} e^{(d+d_0)\left( \vec{k} \cdot \vec{a} + \half \vec{k} \cdot \vec{k} (\e_1 + \e_2) + \ell \e_1 + m \e_2 \right)} g(\vec{k}) Z^{(N)}_{\ell}(\vec{k}) Z^{(S)}_m (\vec{k}) \ . 
\end{align}
Now, we can solve for $Z_n, Z_n^{(N)}, Z_n^{(S)}$ by using the blow-up equation for $d=0, 1, 2$ to obtain
\begin{align}
 Z_n (\vec{a}, m, \e_1, \e_2) = \frac{e^{n(\e_1 + \e_2)} I_n^{(0)} - (e^{n\e_1} + e^{n \e_2}) I_n^{(1)} + I_n^{(2)} }{(1-e^{n\e_1})(1-e^{n\e_2})} \ . 
\end{align}
Since $I_n^{(d)}$ only depends on $Z_m$ with $m<n$ we can determine the instanton partition function recursively. 

We now need to determine the $g(\vec{k})$, but this is also completely determined by the \emph{1-loop part} of the partition function. Therefore we land on a remarkable conclusion: The non-perturbative partition function is completely fixed by the perturbative partition function! Notice that we have arrived at this conclusion not by requiring the perturbative series to be well-defined, as is often done in the resurgence analysis. Instead, we simply demand consistent answer upon deforming the spacetime geometry smoothly. This consistency condition is entirely non-perturbative, that the QFT has to be well-defined regardless of its spacetime. 

In the remainder of this section we will give explicit formulae for 4d and 5d gauge theories. Especially, we will obtain a closed-form expression for the one-instanton partition function for a large class of gauge theories. 

\subsection{Instanton partition function for the 5d theory}
Let us write the perturbative part of the partition function. The classical part is given as
\begin{align}
Z_{\textrm{class}}=\textrm{exp}\Bigg[&\,-\frac{1}{\epsilon_1\epsilon_2}\left(\frac{1}{2}h_{ij} a^i a^j+\frac{1}{6}d_{ijk} a^{i}a^j a^k\right)\Bigg]
\end{align}
where $a_i$ denotes the Coulomb branch parameter and $h_{ij}, d_{ijk}$ denotes the effective gauge couplings and Chern-Simons couplings in the Coulomb branch. 
The 1-loop part of the vector multiplet is given as
\begin{align} \label{eq:1loopvec}
\begin{split}
Z^{V}_{\textrm{1-loop}}&=\textrm{exp}\left[ -\frac{1}{2\epsilon_1\epsilon_2} \sum_{\alpha\in\Delta}\Bigg(\frac{1}{6}(\vec{a}\cdot\vec{\alpha})^3-\frac{\epsilon_1+\epsilon_2}{4}(\vec{a}\cdot\vec{\alpha})^2+\frac{(\epsilon_1+\epsilon_2)^2+\epsilon_1\epsilon_2}{12}(\vec{a}\cdot\vec{\alpha})\Bigg) \right] \\
&\qquad \qquad \times\,\textrm{PE}\left[- \frac{1}{(1-e^{\e_1})(1-e^{\e_2})}\sum_{\alpha\in \Delta}e^{\vec{a}\cdot\vec{\alpha}}\right] \ , 
\end{split}
\end{align}
where $\Delta$ denotes the set of all roots. The 1-loop partition function of the hypermultiplet is given as
\begin{align} \label{eq:1loophyp}
\begin{split}
Z^{H}_{\textrm{1-loop}} &= \exp \Bigg[+\frac{1}{2\epsilon_1\epsilon_2} \sum_{\omega\in R }\Bigg(\frac{1}{6}\left(\vec{a}\cdot\vec{\omega}+m^{\textrm{phy}}_i \right)^3-\frac{\e_1+\e_2}{4}\left(\vec{a}\cdot\vec{\omega}+m^{\textrm{phy}}_i \right)^2 \\
 &\quad -\frac{(\epsilon_1+\epsilon_2)^2+\epsilon_1\epsilon_2}{24}\left(\vec{a}\cdot\vec{\omega}+m^{\textrm{phy}}_i \right)\Bigg)\Bigg] \times 
\textrm{PE} \left[ \frac{e^{m^{\textrm{phy}}_i}}{(1-e^{\e_1})(1-e^{\e_2})} \sum_{\omega\in R}e^{\vec{a}\cdot\vec{\omega}} \right]
\end{split}
\end{align}
where $R$ is the set of all weight vectors in the representation of the hypermultiplet. We define the physical mass parameter as $m^{\textrm{phy}}_i = m_i + \e_+$ and $\e_+ = \frac{\e_1+\e_2}{2}$, since there is an effect of topological twisting that shifts the `mass parameter' $m_i$ by a unit of $SU(2)_R$ rotation \cite{Okuda:2010ke}. 

Now, let us compute the $g(\vec{k})$ which is given by the ratio of the 1-loop part of the partition function. 
It turns out piece of the \eqref{eq:1loopvec}, \eqref{eq:1loophyp} that are not inside the PE eventually cancel out due to $\vec{k} \to -\vec{k}$ and $\vec{a} \to -\vec{a}$ symmetry except for one single term for the hyper \textcolor{red}{(change)}.  We find
% Introducing the symbols $n_1(\vec{k},\vec{w}) \equiv \sum_{\ell=1}^{\vec{k}\cdot \vec{w}} \ell$ and  $n_2(\vec{k},\vec{w}) \equiv \sum_{\ell=1}^{\vec{k}\cdot \vec{w}} (\ell^2-\ell)$ for brevity, we find
\begin{align}
 g(\vec{k})^{V} &= e^{\frac{h^\vee}{2}(\vec{k}\cdot \vec{a})}\prod_{\vec{\a} \in \Delta}  \textrm{PE} \left[  \frac{e^{\vec{a} \cdot \vec{\a}}}{(1-e^{\e_1})(1-e^{\e_2})} - \frac{e^{(\vec{a}+\vec{k}\e_1) \cdot \vec{\a}}}{(1-e^{\e_1})(1-e^{\e_2-\e_1})}- \frac{e^{(\vec{a}+\vec{k}\e_2) \cdot \vec{\a}}}{(1-e^{\e_1- \e_2})(1-e^{\e_2})} \right] ,\nn \\
 g(\vec{k})^{H} &= e^{-\frac{I_2(\mathbf{R})}{4}(\vec{k}\cdot \vec{a})-\frac{I_2(\mathbf{R})}{4}(\vec{k}\cdot \vec{k})m_{\rm phy}} 
 e^{- \frac{I_3(\mathbf{R})}{4} (d_{ijk} a_i k_j k_k ) - \frac{ I_3(\mathbf{R})}{6} \epsilon_+(d_{ijk} k_i k_j k_k )}
 \nn\\&
 \quad\quad\times 
 \prod_{\vec{w} \in R}\textrm{PE} \left[ -\frac{ e^{\vec{a} \cdot \vec{w}+ m^{\textrm{phy}}}}{(1-e^{\e_1})(1-e^{\e_2})} + \frac{e^{(\vec{a}+\vec{k}\e_1) \cdot \vec{w} + m^{\textrm{phy}}}}{(1-e^{\e_1})(1-e^{\e_2-\e_1})} + \frac{e^{(\vec{a}+\vec{k}\e_2) \cdot \vec{w} + m^{\textrm{phy}}}}{(1-e^{\e_1- \e_2})(1-e^{\e_2})} \right]
\end{align}
for the vector and hypermultiplets respectively. Our notation is that $\Tr(T^i T^j) = I_2(\mathbf{R})\delta^{ij}$ and $\Tr(T^i \{T^j,T^k\}) = I_3(\mathbf{R})d_{ijk}$, where $d_{ijk} \neq 0$ only for $G = SU(N)$. The $SU(N)$ generators are normalized such that $I_3(\text{fund}) =1$.
Besides the zero-point energy contribution, the PE of $g(\vec{k})$ can be written as 
% Both of them can be written as 
\begin{align}
%  g(\vec{k}) =  e^{\frac{h^\vee}{2}(\vec{k}\cdot \vec{a})}\prod_{\vec{\a} \in \Lambda}  
\textrm{PE} \left[ e^{\vec{a} \cdot \vec{\a}}A(\vec{k} \cdot \vec{\a}, e^{\e_1} e^{\e_2})\right]\cdot 
 \prod_{i} \prod_{\vec{w} \in R_i} 
%  e^{((\vec{a}\cdot \vec{w}+m^{\rm phy}_i) n_1 + \e_+ n_2}
\textrm{PE} \left[- e^{\vec{a} \cdot \vec{w} + m^{\textrm{phy}}_i}A(\vec{k} \cdot \vec{w}, e^{\e_1} e^{\e_2})\right]
\end{align} 
where $i$ runs over various hypermultiplets in the theory and $p_1 = e^{\e_1}, p_2 = e^{\e_2}$. Here the function $A$ is given as
\begin{align}
\begin{split}
 A(k, p_1, p_2) &= \frac{1}{(1-p_1)(1-p_2)} -  \frac{p_1^k}{(1-p_1)(1-p_2/p_1)} -  \frac{p_2^k}{(1-p_1/p_2)(1-p_2)}  \ . 
\end{split}
\end{align}
We can easily see that $A(k, p_1, p_2)$ vanishes at $k=0, -1$. 
After some work, it is not hard to find that $A(k, p_1, p_2)$ can be written in terms of a finite sum as
\begin{align}
 A(k, p_1, p_2) =
 \begin{cases}
 \displaystyle
 \sum_{m+n \le k-1} p_1^m p_2^n & (k>0) \\
 \displaystyle
 \sum_{m+n \le -k-2} p_1^{-m-1} p_2^{-n-1} & (k<-1) \\
 0 & (k=0, -1)
 \end{cases} . 
\end{align}
Upon taking PE, we obtain
\begin{align}
\mathcal{L}_{k} (x, \e_1, \e_2) \equiv \textrm{PE}\left[ e^{x} A(k, e^{\e_1}, e^{\e_2}) \right] = 
\begin{cases}
 {\displaystyle \prod_{\substack{m, n \ge 0 \\ m+n \le k-1}} \left( 1-e^{x+ m \e_1 +n  \e_2}\right)} & (k > 0) \\
 {\displaystyle \prod_{\substack{m, n \ge 0 \\ m+n \le -k-2}} \left(1-e^{x -(m+1)\e_1 - (n+1)\e_2} \right)} & (k < -1)  \\
 1 & (k=0, -1)
\end{cases}
\end{align}   
Let us state our final answer:
\begin{align}
 Z_n (\vec{a}, \vec{m}^{\textrm{phy}}, \e_1, \e_2) &= \frac{e^{n(\e_1 + \e_2)} I_n^{(0)} - (e^{n\e_1} + e^{n \e_2}) I_n^{(1)} + I_n^{(2)} }{(1-e^{n\e_1})(1-e^{n\e_2})} \ , 
\end{align}
with 
\begin{align}
\begin{split}
I_n^{(d)} = \sum_{\substack{\half \vec{k}\cdot\vec{k} + \ell + m = n \\ \ell, m < n}} & \Bigg( \exp \left[ d\left( \vec{k} \cdot \vec{a} + \half \vec{k} \cdot \vec{k} (\e_1 + \e_2) + \ell \e_1 + m \e_2 \right)\right]  \\& \quad
\times e^{d_0\left( \half \vec{k} \cdot \vec{k} (\e_1 + \e_2) + \ell \e_1 + m \e_2 \right)}  \times e^{- \frac{I_3(\mathbf{R})}{4} (d_{ijk} a_i k_j k_k ) - \frac{ I_3(\mathbf{R})}{6} \epsilon_+(d_{ijk} k_i k_j k_k )}\\
& \quad \times \frac{\displaystyle \prod_{i} \prod_{\vec{w} \in R_i} e^{-\frac{(\vec{k}\cdot \vec{w})^2}{4}m^{\textrm{phy}} } \CL_{\vec{k} \cdot \vec{w}} (\vec{a} \cdot \vec{w} + m^{\textrm{phy}}_i, \e_1, \e_2)}{\displaystyle \prod_{\vec{\a} \in \Lambda}  \CL_{\vec{k} \cdot \vec{\a}}(\vec{a} \cdot \vec{\a}, \e_1, \e_2)  }  \\
&\quad \times Z_{\ell}(\vec{a}+\vec{k}\e_1, \vec{m}^{\textrm{phy}}, \e_1, \e_2-\e_1) Z_m (\vec{a}+\vec{k} \e_2, \vec{m}^{\textrm{phy}}, \e_1-\e_2, \e_2) \Bigg) \ , 
\end{split}
\end{align}
where $i$ runs over all the hypermultiplets. 

\paragraph{One-instanton partition function}
Let us compute the 1-instanton partition function using our formula. In this case, we only take the sum over the long roots having $\vec{k}\cdot \vec{k} = 2$ and $\ell = m = 0$. Therefore, \textcolor{red}{(add the effect of $e^{- \frac{I_3(\mathbf{R})}{4} (d_{ijk} a_i k_j k_k )}$)}
\begin{align}
 I_1^{(d)} = e^{d(\e_1+\e_2)} \sum_{\vec{\g} \in \Delta_l} \frac{e^{d \vec{\g} \cdot \vec{a} } M(\g) }{L(\g)} \ , 
\end{align} 
where $\Delta_l$ is the set of all long roots and
\begin{align}
\begin{split}
 L(\g) &\equiv \prod_{\a \in \Delta} \CL_{\g \cdot \a} (a \cdot \a, \e_1, \e_2) 
  = \prod_{\g \cdot \a = \pm 2} \prod_{\g \cdot \a = 1} \CL_{\g \cdot \a} (a \cdot \a, \e_1, \e_2) \\
 & = (1-e^{a_\g+\e_1})(1-e^{a_\g+\e_2})(1-e^{a_\g}) (1-e^{-a_\g-\e_1-e_2}) \prod_{\g \cdot \a = 1} (1-e^{a_\a }) \ , 
\end{split}
\end{align}
where $a_\g \equiv \vec{a}\cdot \vec{\g}$ with $a_\a \equiv \vec{a}\cdot \vec{\a}$ and 
\begin{align}
 M(\g) \equiv \prod_{\vec{w} \in R} e^{-\frac{(\vec{\g} \cdot \vec{w})^2}{4}m^{\textrm{phy}} }\CL_{\g \cdot w} (a_w, \e_1, \e_2)  = e^{-I(R)m^{\textrm{phy}}/2} \prod_{\vec{w} \in R} \CL_{\g \cdot w} (a_w, \e_1, \e_2) \ , 
\end{align}
with $a_w \equiv \vec{a} \cdot \vec{w}$ and $I(R)$ is the Dynkin index of the representation $R$. 
Therefore, the one-instanton partition function can be written explicitly as 
\begin{align}
\begin{split}
Z_1 &= \frac{e^{\e_1 + \e_2} I_n^{(0)} - (e^{\e_1} + e^{ \e_2}) I_n^{(1)} + I_n^{(2)} }{(1-e^{\e_1})(1-e^{\e_2})} \\
&= \frac{e^{\e_1+\e_2} }{(1-e^{\e_1})(1-e^{\e_2})} \sum_{\g \in \Delta_l} \frac{(1-e^{a_\g+\e_1})(1-e^{a_\g+\e_2}) M(\g) }{L(\g)} \ . 
\end{split}
\end{align}
For the case of pure SYM theory with no matters, $M(\g)=1$ so that 
\begin{align}
Z_1^{\textrm{SYM}} &= \frac{e^{\e_1+\e_2} }{(1-e^{\e_1})(1-e^{\e_2})} \sum_{\g \in \Delta_l} \frac{1}{ (1 - e^{a_\g})(1-e^{-a_\g-\e_1-e_2}) \prod_{\g \cdot \a = 1} (1-e^{a_\a })} \\
&= \frac{e^{\e_1+\e_2} }{(1-e^{\e_1})(1-e^{\e_2})} \sum_{\g \in \Delta_l} \frac{e^{(h^\vee -1)a_\g/2}}{ (e^{a_\g/2} - e^{-a_\g/2})(1-e^{a_\g-\e_1-\e_2}) {\prod_{\g \cdot \a = 1} (e^{a_\a/2}-e^{-a_\a/2 })} } \ , \nn
\end{align}
which is the one derived in \cite{Keller:2011ek, Keller:2012da}. 

For the case with fundamental matters, we find that for the long root $\g$, there are only weights with $\g \cdot w = 0, \pm 1$. Therefore, the $M(\g)$ can be simplified to give
\begin{align}
 M(\g) = \prod_{\g \cdot w = 0, \pm 1} \CL_{\g \cdot w} (a_w, \e_1, \e_2) = \prod_{\g \cdot w = 1} (1-e^{a_w + m^{\textrm{phy} }}) \ . 
\end{align}
The one-instanton partition function is now given as
\begin{align}
Z_1 &= \frac{e^{\e_1+\e_2} }{(1-e^{\e_1})(1-e^{\e_2})} \sum_{\g \in \Delta_l} \frac{\prod_{\g \cdot w = 1} (1-e^{a_w + m^{\textrm{phy} }}) }{ (1 - e^{a_\g})(1-e^{-a_\g-\e_1-e_2}) \prod_{\g \cdot \a = 1} (1-e^{a_\a })} \\
&= \frac{e^{\e_1+\e_2} }{(1-e^{\e_1})(1-e^{\e_2})} \sum_{\g \in \Delta_l} \frac{e^{(h^\vee -1)a_\g/2} \prod_{\g \cdot w = 1} (1-e^{a_w + m^{\textrm{phy} }}) }{ (e^{a_\g/2} - e^{-a_\g/2})(1-e^{a_\g-\e_1-e_2}) {\prod_{\g \cdot \a = 1} (e^{a_\a/2}-e^{-a_\a/2 })} } \ , \nn
\end{align}
which we conjecture to be true for all the hypermultiplets with the representation with $|\g \cdot w| \le 1$ for all $w \in R$. 

\subsection{Instanton partition function for 4d gauge theory}
% Let us discuss the case for the 4d $\CN=2$ theory. In principle, one can simply take the 4d limit of the 5d partition function by shrinking $\beta \to 0$. Instead, let us write down the recursion formula directly to deduce the partition function. 
% The perturbative partition functions for the vector and hypermultiplet are
% \begin{align}
%  Z_{\textrm{vec}}^{\textrm{pert}}(\vec{a}, q) &= \exp \left( - \sum_{\vec \a \in \Delta} \gamma_{\e_1, \e_2}  (\vec{a} \cdot {\vec{\a}}; q )\right) \ , \\
%  Z_{\textrm{hyp}}^{\textrm{pert}} (\vec{a}, m, q)&= \exp \left( \sum_{\vec{w} \in R} \gamma_{\e_1, \e_2} ( \vec{a} \cdot {\vec{w}} - m; q)\right) \ . 
% \end{align}
% Here the gamma function is defined as
% \begin{align}
%  \gamma_{\e_1, \e_2} (x; \Lambda) = \left. \frac{d}{ds} \right|_{s=0} \frac{\Lambda^s}{\Gamma(s)} \int_0^{\infty} \frac{dt}{t} t^s \frac{e^{-ts}}{(e^{\e_1 t} - 1)(e^{\e_2 t} - 1)} \ , 
% \end{align}
% which is formally equivalent to 
% \begin{align}
%  \textrm{log} \left[\prod_{n, m\ge 0} \left( \frac{x - m\e_1 - n \e_2}{\Lambda} \right) \right] \ . 
% \end{align}
% If we write the equation \eqref{eq:blowup} in terms of $Z = Z^{\textrm{pert}} Z^{\textrm{inst}}$, we get
% \begin{align}
%  Z^{\textrm{inst}}(\vec{a}, \e_1, \e_2) = \sum_{\vec{k} \in \Lambda} \frac{Z^{(N), \textrm{pert}}(\vec{k}) Z^{(S), \textrm{pert}}(\vec{k})}{Z^{\textrm{pert}}(\vec{a}, \e_1, \e_2)} Z^{(N), \textrm{inst}}(\vec{k} ) Z^{(S), \textrm{inst}}(\vec{k})
% \end{align}
% and here we omit the dependence on the Coulomb vev and the Omega deformation parameters. 
% Now the factor in the middle can be explicitly worked out. Let us denote the ratio of the perturbative factor as $f(\vec{k}) \equiv Z^{(N), \textrm{pert}} Z^{(S), \textrm{pert}}/Z^{\textrm{pert}}$. 
% Then the ratio of the perturbative factor for the vector multiplet is given as
% \begin{align} \label{eq:1loopvec4d}
% \begin{split}
% f(\vec{k})_{\textrm{vec}} &= \prod_{\vec{\a} \in \Delta} \exp \left( \g_{\e_1, \e_2} (\vec{a}\cdot \vec{\a}) - \g_{\e_1, \e_2 - \e_1}(\vec{a}\cdot \vec{\a} + \vec{k}\cdot\vec{\a} \e_1) -  \g_{\e_1 - \e_2, \e_2 }(\vec{a}\cdot \vec{\a} +  \vec{k}\cdot\vec{\a} \e_2)   \right) \\
%  &= \prod_{\vec{\a} \in \Delta}  \frac{\Lambda^{(\vec{k} \cdot \vec{\a})^2 /2} }{s(-\vec{k}\cdot \vec{\a}, \vec{\a}\cdot \vec{a}, \e_1, \e_2) }
%  = \frac{(\Lambda^{2 h^\vee})^{\vec{k} \cdot \vec{k}/2} }{\prod_{\vec{\a} \in \Delta} \ell^{\vec{k}}_{\vec{\a}} (\vec{a}, \e_1, \e_2) } \ , 
% \end{split}
% \end{align}
% where $h^\vee$ refers to the dual Coxeter number of the gauge group. Notice that the beta function coefficient for the pure YM theory is given by $b_0 = 2h^\vee$ and the instanton parameter is given by $q \equiv \Lambda^{2 h^\vee}$. The other symbols are given as 
% \begin{align}
% \ell^{\vec{k}}_{\vec{\alpha}} (\vec{a}, \e_1, \e_2) &= s(-\vec{k}\cdot\vec{\a}, \vec{a}\cdot\vec{\a}, \e_1, \e_2) \\
% s(k, x, \e_1, \e_2) &= 
% \begin{cases}
%  {\displaystyle \prod_{i, j \ge 0, i+j \le k-1} (x - i \e_1 - j e_2) } & (k > 0) \\
%  {\displaystyle \prod_{i, j \ge 0, i+j \le -k-2} (x + (i+1)\e_1 + (j+1)\e_2)} & (k < -1)  \\
%  1 & (k=0, -1)
% \end{cases}
% \end{align}
% The final identity of \eqref{eq:1loopvec4d} involves a bit of work. This follows from the identity (\cite{Nakajima:2003uh}, App. E.)
% \begin{align}
% \begin{split}
%  \g_{\e_1, \e_2-\e_1}(x+\e_1 k; \Lambda) &+ \g_{\e_1 - \e_2, \e_2}(x+\e_2k; \Lambda) \\
%   & \qquad = \g_{\e_1, \e_2}(x; \Lambda) + \log s(-k, x, \e_1, \e_2) - \frac{k(k-1)}{2} \log \Lambda \ . 
% \end{split}
% \end{align}
% For the hypermultiplets we get, 
% \begin{align}
% f(\vec{k})_{\textrm{hyp}} &=  \prod_{\vec{w} \in R} \exp \Big( -\g_{\e_1, \e_2} (a_{w, m}) + \g_{\e_1, \e_2 - \e_1}(a_{w, m} + k_w \e_1) +  \g_{\e_1 - \e_2, \e_2 }(a_{w, m} + k_w \e_2)   \Big) \nn \\
% &= \prod_{\vec{w} \in R} \Lambda^{-\half k_w^2} s(-{k}_w, a_{w, m}, \e_1, \e_2) 
% = (\Lambda^{- 2 I(R)})^{\half \vec{k} \cdot \vec{k} }\prod_{\vec{w} \in R}  s(-{k}_w, a_{w, m}, \e_1, \e_2) 
% \ , 
% \end{align}
% where we introduced the short-hand notation $k_w = \vec{k}\cdot\vec{w}$, $a_{w, m} = \vec{a} \cdot \vec{w} - m$ and $I(R)$ corresponds to the Dynkin index for the representation $R$. The Dynkin index appears in the beta function coefficients as $b_0 = 2h^\vee - I(R)$ for the hypermultiplets in the representation $R$.  This gives the instanton parameter to be $q \equiv \Lambda^{b_0} = \Lambda^{2h^\vee - I(R)}$. 

% Now, for the SQCD, we obtain the following equation:
% \begin{align}
%   Z^{\textrm{inst}}(\vec{a}, m, \e_1, \e_2) = \sum_{\vec{k} \in \Lambda} f(\vec{k}) Z^{(N), \textrm{inst}}(\vec{k} ) Z^{(S), \textrm{inst}}(\vec{k}) \ , 
% \end{align}
% with
% \begin{align}
%  f(\vec{k}) =  \frac{\displaystyle q^{\half \vec{k} \cdot \vec{k}} \prod_{i} \prod_{\vec{w} \in R_i}  s(-k_w, a_w - m_i, \e_1, \e_2)}{\displaystyle \prod_{\vec{\a} \in \Lambda} s(-k_\a, a_\a, \e_1, \e_2)} \ , 
% \end{align} 
% where $i$ runs over the charged hypermultiplets. Here $q = \Lambda^{2N_c - N_f}$ for the $SU(N_c)$ SQCD with $N_f$ fundamental hypermultiplets. 
% We have checked this expression explicitly for the $SU(2)$ gauge theory with $N_f=0, 1$ hypermultiplets up to the first few order in instanton numbers. 

% Notice that in the Gottsche-Nakajima-Yoshioka \cite{Nakajima:2009qjc, Gottsche:2010ig}, the mass parameters and the instanton parameters (for the 5d) are also shifted when the contribution from North and South poles are computed. This is simply a reflection of the fact that they twist the instanton bundles by the half-Canonical bundle of the $\IC^2$, which shift the mass parameters by $m \to m - \frac{\e_1 + \e_2}{2}$. If we do not twist by this amount, we get a cleaner expression as above. 

%\paragraph{Correlation functions}
%On the blow-up, we have a non-trivial 2-cycle. This allows us to insert the equivariant version of the Donaldson operator, which can be written in terms of the twisted fields as
%\begin{align} \label{eq:muC}
% \left\langle \exp \left[ t \int d^4 x \left( \omega \wedge \phi F + \half \psi \wedge \psi + H F \wedge F \right) \right] \right\rangle \ . 
%\end{align}
%Here $H$ is the moment map for the $U(1)^2_{\e_1, \e_2}$ action so that $dH = \i_{V} \omega$. Upon localization, it is easy to see that it can be written as
%\begin{align}
% \hat{Z}^{\textrm{inst}}(\vec{a}, m, \e_1, \e_2, t) = \sum_{\vec{k} \in \Lambda} e^{ t \left( \vec{k} \cdot \vec{a} + \half \vec{k}\cdot \vec{k} (\e_1 + \e_2) \right) } f(\vec{k}) Z^{(N), \textrm{inst}}(\vec{k}; q e^{t \e_1} ) Z^{(S), \textrm{inst}}(\vec{k} ; q e^{t \e_2}) \ . 
%\end{align} 
%Notice that we shift the instanton parameter by equivariant parameters. This is due to the term $H F\wedge F$ in \eqref{eq:muC} which shifts the instanton parameter by $e^{t(\e_1+\e_2)}$. We need to compensate this part by shifting $q$. 
%The insertion \eqref{eq:muC} inside the exponent has the conserved charge $U=+2$, and the instanton generates $4h^\vee$ which means it counts the number of fermion zero modes. Therefore, when expanding in powers of $t$, 
%\begin{align}
%\hat{Z} = Z + \CO(t^{2h^\vee - 2\sum_R C_2(R)})  \ . 
%\end{align}
%Therefore, when $N_f < 2N_c - 2$ for the $SU(N_c)$ SQCD, we have enough number of relations to fix the instanton partition function. 
%

% \subsection{Blowup Equation for 5d gauge theory} 

% We consider a 5d $\mathcal{N}=1$ gauge theory with a gauge group $G$. It has the Coulomb branch moduli space, parametrized by the vacuum expectation value $\alpha_i\equiv \langle\Phi_{ii}\rangle $ of the vector multiplet scalar $\Phi$. The gauge symmetry $G$ is spontaeously broken to its Abelian subgroup $U(1)^{|G|}$ on the Coulomb branch. The low energy Abelian theory is described by the effective prepotential, which can be written as \cite{Intriligator:1997pq}
% \begin{align}
% 	\mathcal{F}_\text{classical} = \frac{1}{2g^2}h_{ij}\alpha_i \alpha_j + \frac{\kappa}{6}d_{ijk}\alpha_i\alpha_j\alpha_k 
% \end{align}
% at the classical level. It was found in \cite{Nekrasov:2002qd,Nekrasov:2003rj} that the fully quantum corrected prepotential can be obtained from the BPS instanton partition function $\mathcal{Z}$ on $\Omega$-deformed $\mathbf{R}^4 \times S^1$, i.e.,
% \begin{align}
% 	\mathcal{F} = \lim_{\epsilon_{1,2}\rightarrow 0} \epsilon_1 \epsilon_2 \log{\mathcal{Z}} = \mathcal{F}_\text{classical} + \mathcal{F}_\text{quantum}.
% \end{align}

% The BPS partition function $\mathcal{Z}$ is defined with equivariant parameters $\epsilon_{1}, \epsilon_2$, $\alpha_1, \cdots, \alpha_{|G|}$ associated to the $U(1)^2 \times U(1)^{|G|}$ action on the $k$-instanton moduli space $\mathcal{M}_{k,G}$  \cite{Nekrasov:2002qd,Nekrasov:2003rj}. Additional equivariant parameters $m_1, \cdots, m_{|F|}$ can be introduced if a given theory has the flavor symmetry $F$. It takes the form of 
% \begin{align}
%     \mathcal{Z} = \exp{(F_0)} \cdot \mathcal{I}
% \end{align}
% where $\mathcal{I}$ is the Witten index counting the BPS bound states of fundamental particles and/or non-perturbative instanton solitons. More precisely, 
% \begin{align}
% 	\mathcal{I} \equiv \text{Tr}_{\mathcal{H}} \bigg[(-1)^F e^{-\frac{8\pi^2}{g^2} k} e^{-\epsilon_1 (J_1+\frac{R}{2})} e^{-\epsilon_2 (J_2+\frac{R}{2})} \prod_{i=1}^{|G|} e^{-\alpha_i Q_i} \prod_{l=1}^{|F|} e^{-m_l F_l}\bigg]
% \end{align}
% where $(J_1, J_2)$ are the angular momenta associated to the two $\mathbf{R}^2$ planes, $R$ is the Cartan generator of $SU(2)_R$ symmetry, $(Q_1, \cdots, Q_{|G|})$ are the electric charges of $U(1)^{|G|} \subset G$, and $(F_1, \cdots, F_{|F|})$ are the Cartan generators of the flavor symmetry group $F$. 
% We also frequently use the notation $\epsilon_+ \equiv \frac{\epsilon_1 + \epsilon_2}{2}$, $\epsilon_- \equiv \frac{\epsilon_1 - \epsilon_2}{2}$ and $J_l = \frac{J_1 - J_2}{2}$, $J_r = \frac{J_1 + J_2}{2}$, generating self-dual and anti-self-dual rotation inside the $\mathbf{R}^4$.
% The fugacity variables used throughout this paper are 
% \begin{align}
%     p_1 = e^{-\epsilon_1},\ p_2 = e^{-\epsilon_2},\ \omega_i = e^{-\alpha_i},\ y_l = e^{-m_l},\ Q = e^{-8\pi^2  /g^2},\ t = \sqrt{p_1p_2},\  u = \sqrt{p_1/p_2}.
% \end{align}

% Each multiplet




%%%%%%%%%%%%%%%%%%%%%%%%%%%%%%%%%%%%%%%%%%%%%%%%%%%%%%%%%%%%%%%%

\section{Examples} \label{sec:example}

As a remarkable application of the blow-up equation, we construct the instanton partition function of various 5d gauge theories on $\mathbb{C}^2 \times S^1 $. The standard method to compute the multi-instanton correction to the partition function is to employ the ADHM construction of the instanton moduli space \cite{Atiyah:1978ri,Nekrasov:2002qd,Nekrasov:2003rj}, and/or to apply the topological vertex formalism based on the geometric engineering of 5d $\mathcal{N}=1$ gauge theory \cite{Gopakumar:1998jq,Aganagic:2003db}. An alternative approach that comes from the consistency requirement for the blow-up $\hat{\mathbb{C}}^2$ will turn out to be very efficient for bootstrapping the $\mathbb{C}^2 \times S^1 $ partition function of exceptional gauge theories \cite{Keller:2012da, Huang:2017mis, Gu:2018gmy}.

There are multiple and distinct blow-up equations \eqref{eq:blowup5d} at different values of $d$, ranged over $0 \leq d \leq d_{\rm max}$ where $d_{\rm max} \equiv h^\vee 
% -  |\kappa_{\rm eff}| 
- \frac{1}{2}\sum_{i} I(\mathbf{R}_i)$. We need at least three distinct equations to solve for the instanton partition function, thus the bootstrap method is applicable only for gauge theories which satisfy $d_{\rm max} \geq 2$. Here we constrain our discussion to the theories having simple gauge groups, which are classified to have a 5d UV fixed point \cite{Jefferson:2017ahm}. 


There are two 




We will 



The complete list of $\mathcal{N}=1$ gauge theories, satisfying $d_{\rm max} \geq 2$, is displayed in Table~\ref{tbl:infinite-theory}~and~\ref{tbl:exceptional-theory}.
All infinite families of theories, whose rank of the gauge group can be arbitrarily large, are summarized in Table~\ref{tbl:infinite-theory}. There are also finite families of theories, summarized in Table~\ref{tbl:exceptional-theory}, which involve an exceptional gauge group or a spinor hypermultiplet. 

\textcolor{red}{(comparison with IMS bound, etc) (SCFT bound)} 
Some theories in Table~\ref{tbl:exceptional-theory} do not belong to the 

\begin{table}[tbp]
  \centering
  \begin{tabular}{@{}cl@{}}
    \toprule
    Gauge group  & Hypermultiplets    \\ \midrule
    $SU(N)_\kappa$ &x                               \\
    $SO(2N+1)$ & $n_{\mathbf{v}} \leq 2(N-1)$ \\
    $SO(2N)$   & $n_{\mathbf{v}} \leq 2(N-2)$ \\
    $USp(2N)$  & $n_{\bf f} \leq 2N$ \\
    $USp(2N)$  & $n_{\Lambda^2} = 1$ and $n_{\bf f} \leq 2$ \\ \bottomrule
    \end{tabular}
    \caption{asd}
    \label{tbl:infinite-theory}
\end{table}
\begin{table}[tbp]
  \centering
  \begin{tabular}{@{}cl@{}}
    \toprule
    Gauge group  & Hypermultiplets    \\  \midrule
    $SO(14)$  & $n_{\bf s} + n_{\bf c} = 1$ and $n_{\bf v}  \leq 2$ \\
    $SO(13)$  & $n_{\bf s} = 1$ and $n_{\bf v}  \leq 2$ \\
    $SO(12)$  & $n_{\bf s} + n_{\bf c} = 2$ \\
    $SO(12)$  & $n_{\bf s} + n_{\bf c} = 1$ and  $n_{\bf v}  \leq 4$\\
    $SO(11)$  & $n_{\bf s}= 2$ \\
    $SO(11)$  & $n_{\bf s}= 1$ and  $n_{\bf v}  \leq 4$\\
    $SO(10)$  & $n_{\bf s} + n_{\bf c} = 3$ \\
    $SO(10)$  & $n_{\bf s} + n_{\bf c} = 2$ and  $n_{\bf v}  \leq 2$\\
    $SO(10)$  & $n_{\bf s} + n_{\bf c} = 1$ and  $n_{\bf v}  \leq 4$\\
    $SO(9)$  & $n_{\bf s}  = 3$ \\
    $SO(9)$  & $n_{\bf s}  = 2$ and  $n_{\bf v}  \leq 2$\\
    $SO(9)$  & $n_{\bf s} = 1$ and  $n_{\bf v}  \leq 4$\\
    $SO(8)$  & $n_{\bf s} + n_{\bf c} = 4$ \\
    $SO(8)$  & $n_{\bf s} + n_{\bf c} = 3$ and  $n_{\bf v}  =1$\\
    $SO(8)$  & $n_{\bf s} + n_{\bf c} = 2$ and  $n_{\bf v}  \leq 2$\\
    $SO(8)$  & $n_{\bf s} + n_{\bf c} = 1$ and  $n_{\bf v}  \leq 3$\\
    $SO(7)$  & $n_{\bf s} = 4$ \\
    $SO(7)$  & $n_{\bf s} = 3$ and  $n_{\bf v}  =1$\\
    $SO(7)$  & $n_{\bf s} = 2$ and  $n_{\bf v}  \leq 2$\\
    $SO(7)$  & $n_{\bf s} = 1$ and  $n_{\bf v}  \leq 3$\\
    $G_2$  & $n_{\bf 7}\leq 2$ \\
    $F_4$  & $n_{\bf 26}\leq 2$ \\
    $E_6$  & $n_{\bf 27}+ n_{\bf \overline{27}} \leq 3$\\
    $E_7$  & $n_{\bf 56} \leq 2$ \\ 
    $E_8$  & $\varnothing$ \\ \bottomrule
    \end{tabular}
    \caption{asd}
    \label{tbl:exceptional-theory}
\end{table}



% \begin{itemize}
  % \item $G=E_8$: pure YM
  % \item $G=E_7$: $I_{\bf 56} = 12$. $n_{\bf 56} \leq 2$. 
  % \item $G=E_6$: $I_{\bf 27} = I_{\bf \overline{27}} = 6$. $n_{\bf 27}+ n_{\bf \overline{27}} \leq 3$. 
  % \item $G=F_4$: $I_{\bf 26} = 6$. $n_{\bf 26}\leq 2$. 
  % \item $G=G_2$: $I_{\bf 7} = 2$. $n_{\bf 7}\leq 2$. 
  % \item $G=D_N$: $I(spinor) = 2^{N-3}$. $I(vector) = 2$.\\
  %  $G= D_7$: $n_{\bf s} = 1$, $n_{\bf v}  \leq 2$  and $n_{\bf s} = 0$, $n_{\bf v}  \leq 10$. \\
  %  $G=D_6$: $n_{\bf s} = 2$, $n_{\bf v}  = 0$ and $n_{\bf s} = 1$, $n_{\bf v}  \leq 4$ and $n_{\bf s} = 0$, $n_{\bf v}  \leq 8$.\\
  %  $G=D_5$: $n_{\bf s} = 3$, $n_{\bf v}  = 0$ and $n_{\bf s} = 2$, $n_{\bf v}  \leq 2$ and $n_{\bf s} = 1$, $n_{\bf v}  \leq 4$ and $n_{\bf s} = 0$, $n_{\bf v}  \leq 6$\\
  %  $G=D_4$: $n_{\bf s}  + n_{\bf v} \leq 4$
  %  \item $G=B_N$: $I(spinor) = 2^{N-2}$. $I(vector) = 2$.\\
  %  $G= B_6$: $n_{\bf s} = 1$, $n_{\bf v}  \leq 2$  and $n_{\bf s} = 0$, $n_{\bf v}  \leq 10$. \\
  %  $G=B_5$: $n_{\bf s} = 2$, $n_{\bf v}  = 0$ and $n_{\bf s} = 1$, $n_{\bf v}  \leq 4$ and $n_{\bf s} = 0$, $n_{\bf v}  \leq 8$.\\
  %  $G=B_4$: $n_{\bf s} = 3$, $n_{\bf v}  = 0$ and $n_{\bf s} = 2$, $n_{\bf v}  \leq 2$ and $n_{\bf s} = 1$, $n_{\bf v}  \leq 4$ and $n_{\bf s} = 0$, $n_{\bf v}  \leq 6$\\
  %  $G=B_3$: $n_{\bf s}  + n_{\bf v} \leq 4$
  %  \item $G=C_N$:  $I(vector) = 1$. $I(anti) = 2N-2$. $I(\text{rank-3 anti}) = (N-2)(2N -1)$.\\
  %  $G= C_{N}$:  $n_{\bf v} + (2N-2) n_{\Lambda^2} \leq 2N$ . \\
  %  $G= C_3$: $n_{ \Lambda^3} = 1$, $n_{ \Lambda^2} = 0$, $n_{\bf v}  \leq 1$ in addition to the above.\\
  %  ------------------------------------------------------
%    \item $G=A_N$:  $I(fnd) = 1$. $I(anti) = N-2$. $I(sym) = N+2$. $I(\text{rank-3 anti}) = (N - 3) (N - 2)/2$. $I(\text{rank-4 anti}) = (N-4)(N - 3) (N - 2)/3!$. 
%    \begin{align}
%     \kappa_{\rm eff} = \kappa_\text{cl} -\text{sgn}(m) \frac{\mathcal{A}_\mathbf{R_i} }{2}\cdot n_\mathbf{R_i}
%    \end{align}
% \end{itemize}

% \pagebreak



\begin{table}[h]
\centering
\begin{tabular}{|c|c|c|c|}
  \hline
  $G$ & matter & $r_0$ & $d_{\rm max}$\\
  \hline
  $SU(N)_\kappa$ & $N_f\times\boldsymbol{N}$ & $d-N/2-\kappa/2$ & $0\leq d \leq N-|\kappa|-2N_f-1$\textcolor{red}{(?)}\\
  \hline
  $SU(6)_{3}$ & $1\times\boldsymbol{20}$ & $d-6/2-3/2+3/2$ & $1\leq d\leq 6$\\
  \hline
  $SO(7)$ & pure & $d-5/2$ & $0\leq d \leq 5$\\
  \hline
  $SO(7)$ & $1\times\textbf{8}$ & $d-5/2+1/2$ & $0\leq d\leq 4$\\
  \hline
  $SO(7)$ & $1\times\textbf{7}$ & $d-5/2+1\times1/2$ & $0\leq d\leq 4$\\
  \hline
  $SO(7)$ & $2\times\textbf{7}$ & $d-5/2+2\times1/2$ & $0\leq d\leq 3$\\
  \hline
  $G_2$ & pure & $d-4/2$ & $0\leq d \leq 4$\\
  \hline
  $G_2$ & $1\times\textbf{7}$ & $d-4/2+1/2$ & $0\leq d\leq 3$\\
  \hline
  $F_4$ & pure & $d-9/2$ & $0\leq d \leq 9$\\
  \hline
  $F_4$ & $1\times\textbf{26}$ & $d-9/2+1\times 3/2$ & $0\leq d\leq 6$\\
  \hline
  $F_4$ & $2\times\textbf{26}$ & $d-9/2+2\times 3/2$ & $0\leq d\leq 3$\\
  \hline
  \end{tabular}
  \caption{list of theories}
  \label{tbl:list}
\end{table}



\subsection{Known examples}



\subsection{Ki-Hong's note}
\paragraph{Unity Blowup equations}

\paragraph{Instanton partition functions from blowup equations}
From blowup equations one can compute the partition functions as follows. Rewriting the blowup equation as
\begin{align}
1=\sum_{\vec{k}\in\vec{\alpha}^{\lor}}f_{\vec{k}}\,l_{\vec{k}}\frac{Z^{(1)}_{\textrm{inst}}Z^{(2)}_{\textrm{inst}}}{Z_{\textrm{inst}}}
\end{align}
where $f_{\vec{k}}=Z^{(1)}_{\textrm{class}}Z^{(2)}_{\textrm{class}}/Z_{\textrm{class}}$ and $l_{\vec{k}}=Z^{(1)}_{\textrm{1-loop}}Z^{(2)}_{\textrm{1-loop}}/Z_{\textrm{1-loop}}$ with abbreviated notation 
\begin{align}
Z^{(1)}=Z(\epsilon_1,\epsilon_2-\epsilon_1,\vec{a}+\vec{k}\,\epsilon_1,m_i+r_i\,\epsilon_1,m_0+r_0\,\epsilon_1)\nonumber\\
Z^{(2)}=Z(\epsilon_1-\epsilon_2,\epsilon_2,\vec{a}+\vec{k}\,\epsilon_2,m_i+r_i\,\epsilon_2,m_0+r_0\,\epsilon_2)
\end{align} 
Here note that $l_{\vec{k}}$ is independent of $Q=e^{-m_0}$, and $f_{\vec{k}}$ is some overall factor in the order of $Q^{\vec{k}\cdot\vec{k}/2}$. Expanding the equation by instanton fugacity $Q$, then at each $Q^{n}$ level the equation is written by 
\begin{align}
\delta_{n,0}=p_1^{r_0}Z^{(1)}_{n}+p_2^{r_0}Z^{(2)}_{n}-Z_{n}+\sum_{\vec{k}\neq 0}f_{\vec{k},r_0}l_{\vec{k}}\Bigg(\frac{Z^{(1)}_{\textrm{inst}}Z^{(2)}_{\textrm{inst}}}{Z_{\textrm{inst}}}\Bigg)\Bigg|_{O(Q^{n-\vec{k}\cdot\vec{k}/2})}.
\label{eq:be_n}
\end{align}
Since each $Z_{k}$ and $Z^{(1,2)}_{k}$ are independent of $r_0$, one can solve \eqref{eq:be_n} with three blowup equations with same $r_i$'s but different $r_0$'s. 

The blowup equations for instanton partition functions of pure YM theory with generic gauge group were already studied in \cite{Keller:2012da}. They are actually \eqref{eq:bueq_gen} with 
\begin{align}
\vec{r}_a=0,\qquad r_0=d-h^{\lor}/2
\end{align}
where $d=0,\cdots, h^{\lor}$. We extend these blowup equations to the theories with matters based on pure YM blowup equations. If one restrict the cases to $\vec{r}_a=0$, as we explained in the previous section, the $r_i$'s are technically required to be half intergers. Thus we look for the $r_0$'s that provides the correct instanton partition functions by solving \eqref{eq:be_n} while fixing $\vec{r}_a=0$ and $r_i=1/2$. Here are the results.



They were tested by comparing the resulting instanton partition functions with the known results from \cite{Kim:2018gjo}($SO(7)$ and $G_2$) and \cite{DelZotto:2018tcj}($F_4$ with $N_{\boldsymbol{26}}=2$). They were compared numerically, putting random numbers on the fugacities.
Note that matters shift the $r_0$, each by one quarter of their Dynkin indices. It seems to differ from blowup formula for $SU(N)_\kappa+N_f$ instantons, where $r_0$ was affected only by its CS-level $\kappa$. However, one can rewrite the $r_0$ as
\begin{align}
r_0=&\,d-N/2-\left(\kappa+\frac{1}{2}N_f\right)/2+N_f/4\nonumber\\
=&\,d-N/2-\kappa_{\textrm{eff}}/2+N_f\times I_{\textrm{fund}}.
\end{align}
Since fundamental matters shifts the effective CS-level, they cancel their index contributions and consequently the $r_0$ apparently looks independent of matters. 

By above observations, we write the unity blowup equation for generic gauge groups and matter representations.
\begin{align}
Z(\epsilon_1,\epsilon_2,\vec{a},m_i,m_0)=\sum_{\vec{k}\in\vec{\alpha}^{\lor}}&\,Z(\epsilon_1,\epsilon_2-\epsilon_1,\vec{a}+\vec{k}\epsilon_1,m_i+\epsilon_1/2,m_0+r_0\epsilon_1)\nonumber\\
&\times\,Z(\epsilon_1-\epsilon_2,\vec{a}+\vec{k}\epsilon_2,m_i+\epsilon_2/2,m_0+r_0\epsilon_2)
\end{align}
with 
\begin{align}
r_0=d-h^{\lor}/2-\kappa_{\textrm{eff}}/2+N_{\boldsymbol{R}}\times I_{\boldsymbol{R}}.
\end{align}
Here $I_{\boldsymbol{R}}$ is the Dynkin index of $\boldsymbol{R}$ representation.

\paragraph{$SU(6)_3+1\times\boldsymbol{20}$}

As a non-trivial test, we consider the instanton partition function of the $SU(6)_3+\boldsymbol{20}$ whose 5-brane realization was found recently \cite{Hayashi:2019yxj}. Its web-diagram is given as \textcolor{red}{figure}.

\textcolor{red}{(Written before computing the $SU(6)_3+20$ instanton partition function.)}\\
Rather than comparing instanton partition functions directly, we consider an interesting Higgsing procedure. We consider the $SU(3)\times SU(3)\times U(1)\subset SU(6)$ where the $SU(6)$ multiplets are decomposed by
\begin{align}
A_{i\bar{j}}:\,\boldsymbol{35}\longrightarrow&\,(\boldsymbol{8},1)_0\oplus (1,\boldsymbol{8})_0\oplus (\boldsymbol{3},\bar{\boldsymbol{3}})_2\oplus(\bar{\boldsymbol{3}},\boldsymbol{3})_{-2}\oplus(1,1)_0,\nonumber\\
\Phi_{ijk}:\,\boldsymbol{20}\longrightarrow&\,(\boldsymbol{3},\bar{\boldsymbol{3}})_{-1}\oplus(\bar{\boldsymbol{3}},\boldsymbol{3})_1\oplus(1,1)_3\oplus(1,1)_{-3}.
\end{align} 
Here to fit with the web-diagram, we set $\Phi_{156}$ and $\Phi_{234}$ are $(1,1)_3$ and $(1,1)_{-3}$. Once $\Phi_{156}$ and $\Phi_{234}$ get non-zero VEVs, 

 When $a_5=-a_1-a_6$, the web-diagram factorizes to two $SU(3)_3$ whose Coulomb VEVs are $(a_1, a_5, a_6)$ and $(a_2, a_3, a_4)$. In the gauge theory, it can be seen partly from prepotential. The prepotential of $S(6)_3+1\times\boldsymbol{20}$ is
\begin{align}
\mathcal{F}=\frac{1}{2}m_0\sum_{i=1}^{6}a_i^2+\frac{1}{2}\sum_{i=1}^{6}a_i^3+\frac{1}{6}\sum_{i<j}(a_i-a_j)^3-\frac{1}{6}\sum_{1<i<j}(a_1+a_j+a_k)^3
\end{align}
at the Weyl chamber $a_1>\cdots>a_6$. As one sets the Coulomb VEV $a_6=-a_1-a_5$ and $a_4=-a_2-a_3$, one can check
\begin{align}
\mathcal{F}(m_0,a_1,a_2,a_3,a_4,a_5,a_6)=\mathcal{F}_{SU(3)_3}(m_0,a_1,a_5,a_6)+\mathcal{F}_{SU(3)_3}(m_0,a_2,a_3,a_4)
\end{align}
where
\begin{align}
\mathcal{F}_{SU(3)_3}(m_0,a_1,a_2,a_3)=\frac{1}{2}m_0\sum_{i=1}^{3}a_i^2+\frac{1}{2}\sum_{i=1}^{3}a_i^3+\frac{1}{6}\sum_{i<j}(a_i-a_j)^3.
\end{align}
It is Higgsed by 



\subsection{$SU(6)_3$ with a rank-3 antisymmetric hyper}

\textcolor{red}{(describe blow-up computation)}


It was recently found in \cite{Hayashi:2019yxj} that the 5d $SU(6)_3$ gauge theory with a rank-3 antisymmetric hypermultiplet can be engineered from the 5-brane web configuration, depicted in Figure~\ref{fig:SU6-monopole}. Given a web diagram, we utilize  the topological vertex method \cite{Aganagic:2003db,Iqbal:2007ii} to compute all genus topological string amplitudes, which is the logarithm of the 5d Nekrasov partition function on $\Omega$-deformed $\mathbf{R}^4\times S^1$ \cite{Gopakumar:1998jq}. We will check its agreement with the blowup partition function (eqn), providing a supporting evidence to the suggested blow-up equation (eqn).
% We use the notation of \cite{Bao:2013pwa} for the following topological vertex computation. 

By applying the topological vertex method to Figure~\ref{fig:SU6-monopole}, we find that the instanton partition function can be  written as the following sum over all possible 6 Young diagrams:
\begin{align}
  \label{eq:znek-su6}
  Z_{\text{Nek}} 
  =&\, \sum_{(Y_1, \cdots, Y_6)}q^{\sum_{i=1}^6|Y_i|} (-A_1^6)^{|Y_1|}(-A_2^6)^{|Y_2|}(-A_2^2A_3^4)^{|Y_3|}(-A_2^2A_3^2A_4^2)^{|Y_4| + |Y_5|}\nn\\
  &\times f_{Y_1}(g)^5f_{Y_2}(g)^5f_{Y_3}(g)^3f_{Y_4}(g)f_{Y_5}(g)^{-1}f_{Y_6}(g)^{2}Z_{\text{half}}\,(Y_1, Y_2, Y_3, Y_4, Y_5, Y_6)^2. 
  \end{align}
  where the K\"ahler parameters can be identified as 
\begin{align}
A_i = e^{-a_i}, \qquad q = e^{-\frac{8\pi^2}{g^2}}, \qquad g=e^{-\epsilon_-}.
\end{align}
We briefly explain our notation: For a given Young diagram $\mu$, $|\mu|$ denotes the total number of boxes. $\mu_i$ is the number of boxes in the $i$-th row of $\mu$. $\mu^t$ is the transpose of $\mu$. We also use
\begin{align}
f_\mu(g) = (-1)^{|\mu|}g^{\frac{1}{2}({\Vert \mu^t\Vert ^2 - \Vert \mu\Vert ^2})}, \qquad \tilde{Z}_{\lambda}(g) 
= \prod_{(i,j) \in \lambda} {(1 - g^{\lambda_i + \lambda^t_j - i - j +1} )^{-1}}
\end{align}
with $\Vert \mu \Vert^2   = \sum_{i} \mu_i^2$. The factor $Z_{\text{half}}\,(Y_1, Y_2, Y_3, Y_4, Y_5, Y_6)$ involves a single summation over all possible Young diagrams, i.e.,
\begin{align}
Z_{\text{left}}(\vec{Y})= \,
\sum_{Y_0} &( - A_1{}^{-1} A_6{}^{-2})^{|Y_0|} 
g^{\frac{\Vert Y_0^t\Vert ^2+\Vert Y_0\Vert ^2}{2}} \tilde{Z}_{Y_0}(g)^2 f_{Y_0}(g)^2
\ \textstyle\prod_{i=1}^6 g^{\frac{\Vert Y_i\Vert ^2}{2}} \tilde{Z}_{Y_i} (g)
\cr 
& 
\times 
R_{Y_1 Y_6^t}^{-1} (A_1 A_6{}^{-1})\,R_{Y_0 Y_6^t}^{-1} (A_1{}^{-1} A_6{}^{-2})\,R_{Y_1Y_0^t }^{-1} (A_1^2A_6) \cr 
& 
\times  
 \textstyle\prod_{2 \le i <  j \le 5} R_{Y_i Y_j^t}^{-1} (A_i A_j{}^{-1})\ 
 \prod_{i=2}^5 R_{Y_0^t Y_i} (A_1 A_i  A_6) ,
\end{align}
in which we introduce
\begin{align}
R_{\lambda \mu} (Q)=R_{ \mu\lambda} (Q)
&= \text{PE} \left[ - \frac{g}{(1-g)^2} Q \right]
\times N_{\lambda^t \mu} (Q)
\end{align}
with PE representing the Plethystic exponential
 and 
\begin{align}
N_{\lambda \mu} (Q) 
= \prod_{(i,j) \in \lambda} \left( 1 - Q g^{\lambda_i + \mu_j^t -i-j+1} \right)
\prod_{(i,j) \in \mu} \left( 1 - Q g^{-\lambda^t_j - \mu_i + i + j - 1} \right).
\end{align}
We also recall that the Nekrasov partition function is divided into the perturbative partition function $Z_{\text{pert}}$ and the weighted sum of $k$-instanton partition function $Z_k$.
\begin{align}
Z_{\text{Nek}} = Z_{\text{pert}}\Big(1 + \sum_{k=1}^{\infty}q^kZ_k\Big).
\end{align}

The perturbative part of the partition function $Z_{\text{pert}}$ comes from the summand of \eqref{eq:znek-su6} at empty Young diagrams, i.e., $(Y_1, Y_2, Y_3, Y_4, Y_5, Y_6) = (\varnothing,\varnothing,\varnothing,\varnothing,\varnothing,\varnothing)$. It is given by
\begin{align}
  \label{eq:zpert-top}
Z_{\rm pert} 
=& \,
Z_{\text{half}}(\varnothing,\varnothing,\varnothing,\varnothing,\varnothing,\varnothing)^2
\cr
=&\, 
\text{PE} \Biggl[ 
\frac{2g}{(1-g)^2} 
\Bigl( \frac{A_1}{A_6} + \frac{1}{A_1A_6^2} + A_1^2  A_6 
+ \sum_{2 \le i <  j \le 5} \frac{A_i}{A_j}
- \sum_{i=2}^5 A_1 A_i  A_6
\Bigr)
\Biggr]
\cr 
& 
\times \bigg(\sum_{Y_0} ( - A_1{}^{-1} A_6{}^{-2})^{|Y_0|} \ 
g^{\frac{\Vert Y_0^t \Vert ^2+\Vert Y_0\Vert ^2}{2}} \tilde{Z}_{Y_0}(g)^2 f_{Y_0}^2(g)
\cr 
& \qquad \qquad
\textstyle N_{Y_0^t \varnothing }^{-1} (A_1{}^{-1} A_6{}^{-2})
 N^{-1}_{Y_0 \varnothing } (A_1^2  A_6)\prod_{i=2}^5 N_{Y_0 \varnothing } (A_1 A_i  A_6)\bigg)^2 ,
\end{align}
where the last two lines can be  combined into the following closed-form expression:
\begin{align}
  \text{PE} \Biggl[ 
    \frac{2g}{(1-g)^2} 
    \Bigl(\sum_{i=2}^5  \frac{A_1}{A_i} +\sum_{i=2}^5  \frac{A_i}{A_6} - \frac{1}{A_1A_6^2} - A_1^2  A_6 
    - \sum_{2\leq i<j \leq 5} A_1 A_i  A_j + \mathcal{O}(A_1^6)
    \Bigr)
    \Biggr].
\end{align}
So \eqref{eq:zpert-top} is manifestly consistent with the equivariant index  \cite{Shadchin:2005mx} for 5d $SU(6)$ gauge theory with a hypermultiplet in the rank-3 antisymmetric representation $\mathbf{20}$, i.e., 
\begin{align}
  Z_{\rm pert}  = \text{PE} \Biggl[ 
    \frac{2g}{(1-g)^2} 
    \Bigl(\sum_{1\leq i<j \leq 6}  \frac{A_i}{A_j} 
    - \sum_{2\leq i<j \leq 6} A_1 A_i  A_j + \mathcal{O}(A_1^6)
    \Bigr)
    \Biggr].
\end{align}

The 1-instanton partition function $Z_1$ can be obtained from the summands of \eqref{eq:znek-su6} at Young diagrams satisfying $\sum_{i=1}^6 |Y_i|=1$. There are 6 different profiles of Young diagrams. The configuration $|Y_i| = 1$ and $Y_{j\neq i} = \varnothing$ contributes to $Z_{1}$ by
\begin{align}
  + \frac{g}{(1-g)^2} \frac{A_i^6}{\prod_{j \neq i} (A_i - A_j)^2} 
  \Bigl(-A_i \sum_{j\neq i} A_j +  \sum_{j\neq i}\frac{1}{A_j}  - \frac{1}{A_i} + A_i^2
 \Bigr)^2. 
\end{align}
Summing over all six contributions, we find
\begin{align}
  Z_1 = \sum_{i=1}^6 \frac{g}{(1-g)^2} \frac{A_i^6}{\prod_{j \neq i} (A_i - A_j)^2} 
  \Bigl(-A_i \sum_{j\neq i} A_j +  \sum_{j\neq i}\frac{1}{A_j}  - \frac{1}{A_i} + A_i^2
 \Bigr)^2. 
\end{align}
which is in agreement with the blowup partition function (eqn). \textcolor{red}{(overall sign: looking at the GV invariant for single W-boson + single instanton (which should be -2), the topological vertex computation seems to be correct. blowup should have an overall sign issue)}
\pagebreak

%---------  Figure  ---------------%
\begin{figure}[t]
\centering
\includegraphics[width=10cm]{SU6-monopole.jpeg}
\caption{A 5-brane web for $SU(6)_3+1{\bf TAS}$ with massless ${\bf TAS}$.}
\label{fig:SU6-monopole}
\end{figure}
%----------------------------------%

  %---------  Figure  ---------------%
\begin{figure}[t]
\centering
\includegraphics[width=12cm]{SU6-Higgsing.jpeg}
\caption{(a) A Higgsing from from $SU(6)_3+1{\bf TAS}$ to two $SU(3)_3$ theories. (a) Two $SU(3)_3$ theories are painted in blue and in pink respectively. A new decoupled mode emerges.}
\label{fig:SU6-Higgsing}
\end{figure}
%----------------------------------%

%---------  Figure  ---------------%
\begin{figure}[t]
\centering
\includegraphics[width=10cm]{SU6young.jpeg}
\caption{A labeling of Young diagrams assigned to the horizontal lines in Figure \ref{fig:SU6-monopole}.}
\label{fig:SU6young}
\end{figure}
%----------------------------------%



\section{Conclusion} \label{sec:conclusion}


\acknowledgments
This work is supported in part by the UESTC Research Grant A03017023801317 (SSK), the National Research Foundation of Korea (NRF) Grants 2017R1D1A1B06034369 (KL, JS), and 2018R1A2B6004914 (KHL)


\bibliographystyle{JHEP}
\bibliography{ref}

\end{document}

