\documentclass[letterpaper, 11pt]{article}

\usepackage{jheppub}
\usepackage{bm}
\usepackage{graphicx}
\usepackage{epstopdf}
\usepackage{amsmath, amssymb}

\usepackage{amsfonts,amssymb,epsfig,amsmath,mathtools,tabu}
\usepackage{verbatim}

%\usepackage[utf8]{inputenc}
%\usepackage{textcomp,setspace}
\usepackage[inline]{enumitem}

\usepackage{hyperref,caption,subcaption}
\usepackage{xcolor,tikz,graphicx,afterpage}
\usetikzlibrary{shapes.geometric,positioning}
%\usetikzlibrary{cd}


\newcommand{\be}{\begin{eqnarray}}
\newcommand{\ee}{\end{eqnarray}}
\newcommand{\nn}{\nonumber}
\newcommand{\bn}{\begin{enumerate}}
\newcommand{\en}{\end{enumerate}}

%%%%%%%%%%%%%%%%%%% Figures %%%%%%%%%%%%%%%%%%%%%%%%%%%

\newcommand{\fig}[3]{
\begin{figure}
\centerline{\epsfxsize=#1\epsfbox{#2.eps}}
\newcaption{#3. \label{#2}}
\end{figure}
}

%%%%%%%%%%%%% Double line letters using amssymb %%%%%%%%%%%%%%%%

\def\identity{{\rlap{1} \hskip 1.6pt \hbox{1}}}
\def\iden{\identity}

\def\IB{\mathbb{B}}
\def\IC{\mathbb{C}}
\def\ID{\mathbb{D}}
\def\IH{\mathbb{H}}
\def\IM{\mathbb{M}}
\def\IN{\mathbb{N}}
\def\IP{\mathbb{P}}
\def\IR{\mathbb{R}}
\def\IZ{\mathbb{Z}}

%%%%%%%%%%%%%%%% Caligraphic letters %%%%%%%%%%%%%%%%%%

\def\CA{{\cal A}}
\def\CB{{\cal B}}
\def\CC{{\cal C}}
\def\CD{{\cal D}}
\def\CE{{\cal E}}
\def\CF{{\cal F}}
\def\CG{{\cal G}}
\def\CH{{\cal H}}
\def\CI{{\cal I}}
\def\CJ{{\cal J}}
\def\CK{{\cal K}}
\def\CL{{\cal L}}
\def\CM{{\cal M}}
\def\CN{{\cal N}}
\def\CO{{\cal O}}
\def\CP{{\cal P}}
\def\CQ{{\cal Q}}
\def\CR{{\cal R}}
\def\CS{{\cal S}}
\def\CT{{\cal T}}
\def\CU{{\cal U}}
\def\CV{{\cal V}}
\def\CW{{\cal W}}
\def\CX{{\cal X}}
\def\CY{{\cal Y}}
\def\CZ{{\cal Z}}

%%%%%%%%%%%%%%%%%% Greek letters %%%%%%%%%%%%%%%%%%%%%%%%%%%%

\def\a{\alpha}
\def\b{\beta}
\def\g{\gamma}
\def\d{\delta}
\def\e{\epsilon}
\def\ve{\varepsilon}
\def\z{\zeta}
% eta
\def\th{\theta}
\def\vth{\vartheta}
\def\i{\iota}
\def\k{\kappa}
\def\l{\lambda}
\def\m{\mu}
\def\n{\nu}
% xi
% o
% pi
\def\vp{\varpi}
\def\r{\rho}
\def\vr{\varrho}
\def\s{\sigma}
\def\vs{\varsigma}
\def\t{\tau}
\def\u{\upsilon}
% phi
\def\vph{\varphi}
% chi
\def\ch{\chi}
% psi
\def\w{\omega}
%
\def\G{\Gamma}
\def\D{\Delta}
\def\Th{\Theta}
\def\L{\Lambda}
% Xi
% Pi
\def\S{\Sigma}
\def\Y{\Upsilon}
% Phi
% Psi
\def\O{\Omega}


%%%%%%%%%%%%%%%%% Mathematical Symbols %%%%%%%%%%%%%%%%%%%%%%%%%%%%

\def\half{\frac{1}{2}}
\def\thalf{{\textstyle \frac{1}{2}}}
\def\imp{\Longrightarrow}
\def\goto{\rightarrow}
\def\para{\parallel}
\def\vev#1{\langle #1 \rangle}
\def\del{\nabla}
\def\grad{\nabla}
\def\curl{\nabla\times}
\def\div{\nabla\cdot}
\def\p{\partial}
\newcommand{\bra}[1]{\langle{#1}|}
\newcommand{\ket}[1]{|{#1}\rangle}
\def\fslash{\displaystyle{\not}}

%%%%%%%%%%%%%%%%%%%% Normal font in math %%%%%%%%%%%%%%%%%%%%%%%%%%

\def\Tr{{\rm Tr}}
\def\tr{{\rm tr}}
\def\det{{\rm det}}



%%%%%%%%%%%%%%%%%%%%%%%%%%%%%%%%%%%%%%%%%%%%%%%%%%%%%
\title{Exceptional Instantons from Blow-up}

\author[a]{Joonho Kim,}
\author[b]{Sung-Soo Kim,}
\author[c]{Ki-Hong Lee,}
\author[a]{Kimyeong Lee,}
\author[a]{and Jaewon Song}
\affiliation[a]{School of Physics, Korea Institute for Advanced Study, Seoul 02455, Korea}
\affiliation[b]{School of Physics, University of Electronic Science and Technology of China,\\ No.4, Section 2, North Jianshe Road, Chengdu, Sichuan 610054, China}
\affiliation[c]{Department of Physics and Astronomy \& Center for Theoretical Physics\\ Seoul National University, Seoul 08826, Korea}
\emailAdd{joonhokim@kias.re.kr}
\emailAdd{sungsoo.kim@uestc.edu.cn}
\emailAdd{khlee11812@gmail.com}
\emailAdd{klee@kias.re.kr}
\emailAdd{jsong@kias.re.kr}

\abstract{
    
}


\preprint{KIAS-P19???, SNUTP19-???}

%%%%%%%%%%%%%%%%%%%%%%%%%%%%%%%%%%%%%%%%%%%%%%%%%%%
\begin{document}
\maketitle

\section{Introduction} \label{sec:intro}

%%%%%%%%%%%%%%%%%%%%%%%%%%%%%%%%%%%%%%%%%%%%%%%%%%%%%%%%%%%%%%%%

\section{Blowup Equation} \label{sec:blowup}

We consider a 5d $\mathcal{N}=1$ gauge theory with a gauge group $G$. It has the Coulomb branch moduli space, parametrized by the vacuum expectation value $\alpha_i\equiv \langle\Phi_{ii}\rangle $ of the vector multiplet scalar $\Phi$. The gauge symmetry $G$ is spontaeously broken to its Abelian subgroup $U(1)^{|G|}$ on the Coulomb branch. The low energy Abelian theory is described by the effective prepotential, which can be written as \cite{Intriligator:1997pq}
\begin{align}
	\mathcal{F}_\text{classical} = \frac{1}{2g^2}h_{ij}\alpha_i \alpha_j + \frac{\kappa}{6}d_{ijk}\alpha_i\alpha_j\alpha_k 
\end{align}
at the classical level. It was found in \cite{Nekrasov:2002qd,Nekrasov:2003rj} that the fully quantum corrected prepotential can be obtained from the BPS instanton partition function $\mathcal{Z}$ on $\Omega$-deformed $\mathbf{R}^4 \times S^1$, i.e.,
\begin{align}
	\mathcal{F} = \lim_{\epsilon_{1,2}\rightarrow 0} \epsilon_1 \epsilon_2 \log{\mathcal{Z}} = \mathcal{F}_\text{classical} + \mathcal{F}_\text{quantum}.
\end{align}

The BPS partition function $\mathcal{Z}$ is defined with equivariant parameters $\epsilon_{1}, \epsilon_2$, $\alpha_1, \cdots, \alpha_{|G|}$ associated to the $U(1)^2 \times U(1)^{|G|}$ action on the $k$-instanton moduli space $\mathcal{M}_{k,G}$  \cite{Nekrasov:2002qd,Nekrasov:2003rj}. Additional equivariant parameters $m_1, \cdots, m_{|F|}$ can be introduced if a given theory has the flavor symmetry $F$. It takes the form of 
\begin{align}
    \mathcal{Z} = \exp{(F_0)} \cdot \mathcal{I}
\end{align}
where $\mathcal{I}$ is the Witten index counting the BPS bound states of fundamental particles and/or non-perturbative instanton solitons. More precisely, 
\begin{align}
	\mathcal{I} \equiv \text{Tr}_{\mathcal{H}} \bigg[(-1)^F e^{-\frac{8\pi^2}{g^2} k} e^{-\epsilon_1 (J_1+\frac{R}{2})} e^{-\epsilon_2 (J_2+\frac{R}{2})} \prod_{i=1}^{|G|} e^{-\alpha_i Q_i} \prod_{l=1}^{|F|} e^{-m_l F_l}\bigg]
\end{align}
where $(J_1, J_2)$ are the angular momenta associated to the two $\mathbf{R}^2$ planes, $R$ is the Cartan generator of $SU(2)_R$ symmetry, $(Q_1, \cdots, Q_{|G|})$ are the electric charges of $U(1)^{|G|} \subset G$, and $(F_1, \cdots, F_{|F|})$ are the Cartan generators of the flavor symmetry group $F$. 
We also frequently use the notation $\epsilon_+ \equiv \frac{\epsilon_1 + \epsilon_2}{2}$, $\epsilon_- \equiv \frac{\epsilon_1 - \epsilon_2}{2}$ and $J_l = \frac{J_1 - J_2}{2}$, $J_r = \frac{J_1 + J_2}{2}$, generating self-dual and anti-self-dual rotation inside the $\mathbf{R}^4$.
The fugacity variables used throughout this paper are 
\begin{align}
    p_1 = e^{-\epsilon_1},\ p_2 = e^{-\epsilon_2},\ \omega_i = e^{-\alpha_i},\ y_l = e^{-m_l},\ Q = e^{-8\pi^2  /g^2},\ t = \sqrt{p_1p_2},\  u = \sqrt{p_1/p_2}.
\end{align}

Each multiplet




%%%%%%%%%%%%%%%%%%%%%%%%%%%%%%%%%%%%%%%%%%%%%%%%%%%%%%%%%%%%%%%%

\section{Examples} \label{sec:example}

\subsection{Ki-Hong's note}
\paragraph{Unity Blowup equations}
The partition functions of generic 5d $\mathcal{N}=1$ gauge theories with hypermultiplets in $R$-representation in the Coulomb branch consist of classical action term, 1-loop term, and instanton partition functions.
\begin{align}
Z(\epsilon_1,\epsilon_2,\vec{a},m_i,m_0)=Z_{\textrm{class}}(\epsilon_1,\epsilon_2,\vec{a},m_0)\,Z_{\textrm{1-loop}}(\epsilon_1,\epsilon_2,\vec{a},m_i)\,Z_{\textrm{inst}}(\epsilon_1,\epsilon_2,\vec{a},m_i,m_0)
\label{eq:bueq_gen}
\end{align}
where
\begin{align}
Z_{\textrm{class}}=\textrm{exp}\Bigg[&\,-\frac{1}{\epsilon_1\epsilon_2}\left(\frac{1}{2}h_{ij}\phi^i\phi^j+\frac{1}{6}d_{ijk}\phi^{i}\phi^j\phi^k\right)\Bigg]\nonumber\\
Z_{\textrm{1-loop}}=\textrm{exp}\Bigg[&\,-\frac{1}{2\epsilon_1\epsilon_2}\Bigg(\sum_{\alpha\in\textrm{roots}}\Bigg(\frac{1}{6}(\vec{a}\cdot\vec{\alpha})^3-\frac{1}{4}(\epsilon_1+\epsilon_2)(\vec{a}\cdot\vec{\alpha})^2+\frac{1}{12}((\epsilon_1+\epsilon_2)^2+\epsilon_1\epsilon_2)(\vec{a}\cdot\vec{\alpha})\Bigg)\nonumber\\
&\,+\sum_{\omega\in\rho(R)}\Bigg(\frac{1}{6}\Big(\vec{a}\cdot\vec{\omega}+m_i+\frac{\epsilon_1+\epsilon_2}{2}\Big)^3-\frac{\epsilon_1+\epsilon_2}{4}\Big(\vec{a}\cdot\vec{\omega}+m_i+\frac{\epsilon_1+\epsilon_2}{2}\Big)^2\nonumber\\
&\,\qquad\qquad-\frac{(\epsilon_1+\epsilon_2)^2+\epsilon_1\epsilon_2}{24}\Big(\vec{a}\cdot\vec{\omega}+m_i+\frac{\epsilon_1+\epsilon_2}{2}\Big)\Bigg)\Bigg]\nonumber\\
\times\,\textrm{PE}\Bigg[&\,\frac{1}{(1-p_1)(1-p_2)}\Bigg(-\sum_{\alpha\in\textrm{roots}}e^{\vec{a}\cdot\vec{\alpha}}+p_1^{1/2}p_2^{1/2}y_i\sum_{\omega\in \rho(R)}e^{\vec{a}\cdot\vec{\omega}}\Bigg)\Bigg].
\end{align}
Here $\vec{a}$ are Coulomb VEVs and $p_{1,2}=e^{\epsilon_{1,2}}$, $y_i=e^{m_i}$. Note that the normal exponential term saturates the zero-point energy of pletheystic exponential terms.\footnote{Technically, instead of considering this 1-loop prepotential terms, I inserted overall factors to the $l_{\vec{k}}=Z_{\textrm{1-loop}}^{(1)}Z_{\textrm{1-loop}}^{(1)}/Z_{\textrm{1-loop}}$ so that it is written by Sinh terms.} 

The partition function satisfies so-called ``Unity blowup equation" 
\begin{align}
Z(\epsilon_1,\epsilon_2,\vec{a},m_i, m_0)=\sum_{\vec{k}\in\vec{\alpha}^{\lor}}&\,Z(\epsilon_1,\epsilon_2-\epsilon_1,\vec{a}+(\vec{k}+\vec{r}_a)\,\epsilon_1,m_i+r_i\,\epsilon_1,m_0+r_0\,\epsilon_1)\nonumber\\
&\times\, Z(\epsilon_1-\epsilon_2,\epsilon_2,\vec{a}+(\vec{k}+\vec{r}_a)\,\epsilon_2,m_i+r_i\,\epsilon_2,m_0+r_0\,\epsilon_2)
\end{align}
for certain $\vec{r}_a,\, r_i,\,r_0$'s.
Here $\vec{\alpha}^{\lor}$ is the coroot lattice where the long root is normalized to have norm 2. The $r_i$'s and $r_0$ are some numbers specifying the blowup equations.

Technically $r_i$'s are constrained to be half integers since, for each single letter 1-loop partition functions
\begin{align}
Z_{i,\vec{\omega}}=\textrm{PE}\Bigg[\frac{p_1^{1/2}p_2^{1/2}}{(1-p_1)(1-p_2)}y_i\,e^{\vec{a}\cdot\vec{\omega}}\Bigg],
\end{align}
the ratio between shifted ones and unshifted one is
\begin{align}
l^{\vec{k}}_{i,\vec{\omega}}=&\,Z^{(1)}_{i,\vec{\omega}}Z^{(2)}_{i,\vec{\omega}}/Z_{i,\vec{\omega}}\nonumber\\
=&\,\textrm{PE}\Bigg[\frac{p_1^{r_i}p_2^{1/2}y_i}{(1-p_1)(1-p_2/p_1)}p_1^{\vec{k}\cdot\vec{\omega}}e^{\vec{a}\cdot\vec{\omega}}+\frac{p_1^{1/2}p_2^{r_i}y_i}{(1-p_1/p_2)(1-p_2)}p_2^{\vec{k}\cdot\vec{\omega}}e^{\vec{a}\cdot\vec{\omega}}-\frac{p_1^{1/2}p_2^{1/2}y_i}{(1-p_1)(1-p_2)}e^{\vec{a}\cdot\vec{\omega}}\Bigg]\nonumber\\
=&\,\textrm{PE}\Bigg[\frac{p_1^{1/2}p_2^{1/2}y_i}{(1-p_1)(1-p_2)(p_1-p_2)}e^{\vec{a}\cdot\vec{\omega}}\Big((1-p_2)p_1^{\vec{k}\cdot\vec{\omega}+r_i+1/2}-(1-p_1)p_2^{\vec{k}\cdot\vec{\omega}+r_i+1/2}\Big)-(p_1-p_2)\Bigg]
\end{align}
For the $l^{\vec{k}}_{i,\vec{\omega}}$ to be finite rational function, the plethystic exponent must be finite series. It can be satisfied only when $r_i$ is a half integer.

\paragraph{Instanton partition functions from blowup equations}
From blowup equations one can compute the partition functions as follows. Rewriting the blowup equation as
\begin{align}
1=\sum_{\vec{k}\in\vec{\alpha}^{\lor}}f_{\vec{k}}\,l_{\vec{k}}\frac{Z^{(1)}_{\textrm{inst}}Z^{(2)}_{\textrm{inst}}}{Z_{\textrm{inst}}}
\end{align}
where $f_{\vec{k}}=Z^{(1)}_{\textrm{class}}Z^{(2)}_{\textrm{class}}/Z_{\textrm{class}}$ and $l_{\vec{k}}=Z^{(1)}_{\textrm{1-loop}}Z^{(2)}_{\textrm{1-loop}}/Z_{\textrm{1-loop}}$ with abbreviated notation 
\begin{align}
Z^{(1)}=Z(\epsilon_1,\epsilon_2-\epsilon_1,\vec{a}+\vec{k}\,\epsilon_1,m_i+r_i\,\epsilon_1,m_0+r_0\,\epsilon_1)\nonumber\\
Z^{(2)}=Z(\epsilon_1-\epsilon_2,\epsilon_2,\vec{a}+\vec{k}\,\epsilon_2,m_i+r_i\,\epsilon_2,m_0+r_0\,\epsilon_2)
\end{align} 
Here note that $l_{\vec{k}}$ is independent of $Q=e^{-m_0}$, and $f_{\vec{k}}$ is some overall factor in the order of $Q^{\vec{k}\cdot\vec{k}/2}$. Expanding the equation by instanton fugacity $Q$, then at each $Q^{n}$ level the equation is written by 
\begin{align}
\delta_{n,0}=p_1^{r_0}Z^{(1)}_{n}+p_2^{r_0}Z^{(2)}_{n}-Z_{n}+\sum_{\vec{k}\neq 0}f_{\vec{k},r_0}l_{\vec{k}}\Bigg(\frac{Z^{(1)}_{\textrm{inst}}Z^{(2)}_{\textrm{inst}}}{Z_{\textrm{inst}}}\Bigg)\Bigg|_{O(Q^{n-\vec{k}\cdot\vec{k}/2})}.
\label{eq:be_n}
\end{align}
Since each $Z_{k}$ and $Z^{(1,2)}_{k}$ are independent of $r_0$, one can solve \eqref{eq:be_n} with three blowup equations with same $r_i$'s but different $r_0$'s. 

The blowup equations for instanton partition functions of pure YM theory with generic gauge group were already studied in \cite{Keller:2012da}. They are actually \eqref{eq:bueq_gen} with 
\begin{align}
\vec{r}_a=0,\qquad r_0=d-h^{\lor}/2
\end{align}
where $d=0,\cdots, h^{\lor}$. We extend these blowup equations to the theories with matters based on pure YM blowup equations. If one restrict the cases to $\vec{r}_a=0$, as we explained in the previous section, the $r_i$'s are technically required to be half intergers. Thus we look for the $r_0$'s that provides the correct instanton partition functions by solving \eqref{eq:be_n} while fixing $\vec{r}_a=0$ and $r_i=1/2$. Here are the results.

\begin{tabular}{|c|c|c|c|}
\hline
$G$ & matter & $r_0$ & $d$\\
\hline
$SU(N)_\kappa$ & $N_f\times\boldsymbol{N}$ & $d-N/2-\kappa/2$ & $0\leq d \leq N-|\kappa|-2N_f-1$\textcolor{red}{(?)}\\
\hline
$SU(6)_{3}$ & $1\times\boldsymbol{20}$ & $d-6/2-3/2+3/2$ & $1\leq d\leq 6$\\
\hline
$SO(7)$ & pure & $d-5/2$ & $0\leq d \leq 5$\\
\hline
$SO(7)$ & $1\times\textbf{8}$ & $d-5/2+1/2$ & $0\leq d\leq 4$\\
\hline
$SO(7)$ & $1\times\textbf{7}$ & $d-5/2+1\times1/2$ & $0\leq d\leq 4$\\
\hline
$SO(7)$ & $2\times\textbf{7}$ & $d-5/2+2\times1/2$ & $0\leq d\leq 3$\\
\hline
$G_2$ & pure & $d-4/2$ & $0\leq d \leq 4$\\
\hline
$G_2$ & $1\times\textbf{7}$ & $d-4/2+1/2$ & $0\leq d\leq 3$\\
\hline
$F_4$ & pure & $d-9/2$ & $0\leq d \leq 9$\\
\hline
$F_4$ & $1\times\textbf{26}$ & $d-9/2+1\times 3/2$ & $0\leq d\leq 6$\\
\hline
$F_4$ & $2\times\textbf{26}$ & $d-9/2+2\times 3/2$ & $0\leq d\leq 3$\\
\hline
\end{tabular}


They were tested by comparing the resulting instanton partition functions with the known results from \cite{Kim:2018gjo}($SO(7)$ and $G_2$) and \cite{DelZotto:2018tcj}($F_4$ with $N_{\boldsymbol{26}}=2$). They were compared numerically, putting random numbers on the fugacities.
Note that matters shift the $r_0$, each by one quarter of their Dynkin indices. It seems to differ from blowup formula for $SU(N)_\kappa+N_f$ instantons, where $r_0$ was affected only by its CS-level $\kappa$. However, one can rewrite the $r_0$ as
\begin{align}
r_0=&\,d-N/2-\left(\kappa+\frac{1}{2}N_f\right)/2+N_f/4\nonumber\\
=&\,d-N/2-\kappa_{\textrm{eff}}/2+N_f\times I_{\textrm{fund}}.
\end{align}
Since fundamental matters shifts the effective CS-level, they cancel their index contributions and consequently the $r_0$ apparently looks independent of matters. 

By above observations, we write the unity blowup equation for generic gauge groups and matter representations.
\begin{align}
Z(\epsilon_1,\epsilon_2,\vec{a},m_i,m_0)=\sum_{\vec{k}\in\vec{\alpha}^{\lor}}&\,Z(\epsilon_1,\epsilon_2-\epsilon_1,\vec{a}+\vec{k}\epsilon_1,m_i+\epsilon_1/2,m_0+r_0\epsilon_1)\nonumber\\
&\times\,Z(\epsilon_1-\epsilon_2,\vec{a}+\vec{k}\epsilon_2,m_i+\epsilon_2/2,m_0+r_0\epsilon_2)
\end{align}
with 
\begin{align}
r_0=d-h^{\lor}/2-\kappa_{\textrm{eff}}/2+N_{\boldsymbol{R}}\times I_{\boldsymbol{R}}.
\end{align}
Here $I_{\boldsymbol{R}}$ is the Dynkin index of $\boldsymbol{R}$ representation.

\paragraph{$SU(6)_3+1\times\boldsymbol{20}$}

As a non-trivial test, we consider the instanton partition function of the $SU(6)_3+\boldsymbol{20}$ whose 5-brane realization was found recently \cite{Hayashi:2019yxj}. Its web-diagram is given as \textcolor{red}{figure}.

\textcolor{red}{(Written before computing the $SU(6)_3+20$ instanton partition function.)}\\
Rather than comparing instanton partition functions directly, we consider an interesting Higgsing procedure. We consider the $SU(3)\times SU(3)\times U(1)\subset SU(6)$ where the $SU(6)$ multiplets are decomposed by
\begin{align}
A_{i\bar{j}}:\,\boldsymbol{35}\longrightarrow&\,(\boldsymbol{8},1)_0\oplus (1,\boldsymbol{8})_0\oplus (\boldsymbol{3},\bar{\boldsymbol{3}})_2\oplus(\bar{\boldsymbol{3}},\boldsymbol{3})_{-2}\oplus(1,1)_0,\nonumber\\
\Phi_{ijk}:\,\boldsymbol{20}\longrightarrow&\,(\boldsymbol{3},\bar{\boldsymbol{3}})_{-1}\oplus(\bar{\boldsymbol{3}},\boldsymbol{3})_1\oplus(1,1)_3\oplus(1,1)_{-3}.
\end{align} 
Here to fit with the web-diagram, we set $\Phi_{156}$ and $\Phi_{234}$ are $(1,1)_3$ and $(1,1)_{-3}$. Once $\Phi_{156}$ and $\Phi_{234}$ get non-zero VEVs, 

 When $a_5=-a_1-a_6$, the web-diagram factorizes to two $SU(3)_3$ whose Coulomb VEVs are $(a_1, a_5, a_6)$ and $(a_2, a_3, a_4)$. In the gauge theory, it can be seen partly from prepotential. The prepotential of $S(6)_3+1\times\boldsymbol{20}$ is
\begin{align}
\mathcal{F}=\frac{1}{2}m_0\sum_{i=1}^{6}a_i^2+\frac{1}{2}\sum_{i=1}^{6}a_i^3+\frac{1}{6}\sum_{i<j}(a_i-a_j)^3-\frac{1}{6}\sum_{1<i<j}(a_1+a_j+a_k)^3
\end{align}
at the Weyl chamber $a_1>\cdots>a_6$. As one sets the Coulomb VEV $a_6=-a_1-a_5$ and $a_4=-a_2-a_3$, one can check
\begin{align}
\mathcal{F}(m_0,a_1,a_2,a_3,a_4,a_5,a_6)=\mathcal{F}_{SU(3)_3}(m_0,a_1,a_5,a_6)+\mathcal{F}_{SU(3)_3}(m_0,a_2,a_3,a_4)
\end{align}
where
\begin{align}
\mathcal{F}_{SU(3)_3}(m_0,a_1,a_2,a_3)=\frac{1}{2}m_0\sum_{i=1}^{3}a_i^2+\frac{1}{2}\sum_{i=1}^{3}a_i^3+\frac{1}{6}\sum_{i<j}(a_i-a_j)^3.
\end{align}
It is Higgsed by 

%%%%%%%%%%%%%%%%%%%%%%%%%%%%%%%%%%%%%%%%%%%%%%%%%%%%%%%%%%%%%%%%

\section{Conclusion} \label{sec:conclusion}


\acknowledgments
This work is supported in part by the UESTC Research Grant A03017023801317 (SSK), the National Research Foundation of Korea (NRF) Grants 2017R1D1A1B06034369 (KL, JS), and 2018R1A2B6004914 (KHL)


\bibliographystyle{JHEP}
\bibliography{ref}

\end{document}

